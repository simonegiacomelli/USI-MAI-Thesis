%v1.16

\documentclass[11pt, mscthesis]{usiinfthesis}
\usepackage[utf8]{inputenc}
\usepackage[T1]{fontenc} % Output font encoding for international characters
\usepackage{svg}
\usepackage[english]{babel}
\usepackage{lipsum}
\usepackage{listings}
\usepackage{graphicx}
\usepackage{subfig}
\lstset{columns=fixed, basicstyle=\ttfamily, basewidth=0.5em}

% \RequirePackage{scrhack} % Loads fixes for various packages

\graphicspath{{./images/}{./images2/}}
\DeclareGraphicsExtensions{*.svg,.png,.pdf}

\newcommand{\vect}[1]{\boldsymbol{#1}}
%%%%%
\makeatletter
\newcommand{\labeltextbf}[3][]{%
    \@bsphack%
    \csname phantomsection\endcsname% in case hyperref is used
    \def\tst{#1}%
    \def\labelmarkup{\textbf}% How to markup the label itself
    %\def\refmarkup{\labelmarkup}% How to markup the reference
    \def\refmarkup{}%
    \ifx\tst\empty\def\@currentlabel{\refmarkup{#2}}{\label{#3}}%
    \else\def\@currentlabel{\refmarkup{#1}}{\label{#3}}\fi%
    \@esphack%
    \labelmarkup{#2}% visible printed text.
}
\makeatother

%%%%%
\lstdefinelanguage{algebra}
{morekeywords={import,sort,constructors,observers,transformers,axioms,if,
else,end},
sensitive=false,
morecomment=[l]{//s},
}
\DeclareUnicodeCharacter{0301}{*************************************}


\usepackage{subfiles}
\usepackage{biblatex}
\addbibresource{biblio.bib}

\title{Using Deep Learning to Identify Technical Debt} %compulsory
\specialization{Major in Artificial Intelligence}%optional
\subtitle{Levereging Self Admitted Technical Debt} %optional 
\author{Simone Giacomelli} %compulsory
\begin{committee}
\advisor{Prof. Dr.}{Gabriele Bavota}{} %compulsory
\coadvisor{Dr.}{Csaba Nagy}{}{} %optional
\end{committee}
\Day{1} %compulsory
\Month{January} %compulsory
\Year{2021} %compulsory, put only the year
\place{Lugano} %compulsory

\dedication{To the part of you that will always\\learn new things and strive to be better}

%\openepigraph{Someone said \dots}{Someone} %optional

%\makeindex %optional, also comment out \theindex at the end

\begin{document}

\maketitle %generates the titlepage, this is FIXED

\frontmatter %generates the frontmatter, this is FIXED

\begin{abstract}
% Composed of four parts: \\
% a: context\\
% b: problem\\
% c: proposed solution\\
% d: results\\
% \\
% max 400 words.
The `technical debt' metaphor describes a misalignment between the appropriate solution to a problem and the non-optimal actual implementation. Software developers accumulate technical debt when they choose speed at the expense of quality and correctness. As the debt metaphor suggests, there are multiple elements involved; two of them are: gain and interest. The gain is a shorter time to market and lower initial costs. The interest is all the additional effort caused by choosing the easy but limited solution, i.e. it's harder to make changes on suboptimal code implementation. When the debt is not paid in a timely manner the interest can crush the evolution capabilities of the project itself. Researcher agrees that it's important to track, manage and repay technical debt to avoid dire consequences.

Developers, while introducing technical debt, can decide to leave a source code comment to document what they are doing; this is called self-admitted technical debt (SATD). A SATD is an acknowledgement that something needs to be fixed.

In this thesis we propose a technical debt classifier fueled by the SATD harvested from 245,243 GitHub Java projects. We detail the process of the creation of the dataset. We explain how to extract features from a source snippet; these features are learned by the network and, also using an attention mechanism, encoded in a fixed-length vector. The last layer of the deep learning model finally classifies the snippet as TD-free or TD-affected.  

In this empirical study we assess, using quantitative and qualitative research, the accuracy of the classifier and analyze how the prediction confidence influences it.

The results on multiple datasets show a precision ranging between 67\% and 71\%, and a F1-score between 61\% and 66\%. Our quantitative analysis on the prediction confidence score shows that we can significantly increase the precision at the expanse of lowering the recall. 
We explore quantitative and qualitative findings in both correct and incorrect prediction; we show patterns successfully learned by the network (empty exception block, magic constant and return null) and we also show why specific predictions fail. We conducted an experiment to optimize the hyperparameters of the model using a distributed computing grid search.


% The results on the confidence level analysis show interesting results: when filtering for a confidence level greater than 0.9 we measured a gain on precision, from 71\% to 78\%; the excluded samples shows their effect also on the recall that goes from 62\% down to 52\%. When using a dataset with bigger code snippets and confidence greater than 0.9 we discard 98\% of the test predictions; however the precision is high as 99\% and the recall drops to 10\%, both due to the (correct and incorrect) discarded predictions.


\end{abstract}


\begin{acknowledgements}
I would like to thank my advisors Prof. Dr. Gabriele Bavota and Dr. Csaba Nagy. They helped me more than they can imagine. Their insightful comments and professionalism contributed immensely to make this thesis an amazing and rewarding project.
\\
\\
A huge thanks to all the people that make my university tick. I'm grateful to all the staff of Universit\`a della Svizzera Italiana, you inspired me on the journey that led me to join the Academic Senate and the Council of the Faculty of Informatics as student representative. I met wonderful people and you all made it an invaluable learning experience. You will always have a special place in my heart. USI is a young and beautiful university, I hope you are enjoying it like I did.

%thanks family

%thanks friends

\end{acknowledgements}

\tableofcontents 
\listoffigures %optional
\listoftables %optional

\mainmatter

\chapter{Introduction}
Technical Debt (TD) is a metaphor that has its roots in the financial field. In software engineering, a TD is contracted when a workaround or shortcut is taken during code implementation.
Choosing an easy and quick solution over a slower and more appropriate one can save time to deliver the artifact faster in the short run. However, it has many downsides in the long run. Many studies have shown that further work on the affected parts will be more expensive and time consuming than work on clean and healthy code \cite{tom2013exploration} \cite{allman2012managing} \cite{guo2016exploring} \cite{besker2018embracing} \cite{cunningham1992wycash}.
There are also indirect effects to contemplate: the software could misbehave in the operating domain; unexpected output or wrong behavior can cause damages and increased costs \cite{guo2011tracking}.
% \\
% TD do not only appear in software projects but can also be found in many layers of a technology stack: for example, delaying an hardware upgrade or a maintenance can give an immediate benefit of less downtime or financial savings, but an increased cost for future unexpected downtime or failures \cite{allman2012managing}.
%
Developers or managers can accumulate TD, knowingly or unknowingly, because of strict deadlines, limited resources available or just plain laziness \cite{hinsen2015technical,allman2012managing}. Cunningham, who coined the technical debt metaphor, writes: `A little debt speeds development so long as it is paid back promptly with a rewrite' \cite{cunningham1992wycash}. Cunningham implies that one could benefit from a small amount of TD but it should be paid back as soon as possible. 
%
%TD can arise from intentional and unintentional decision, e.g. an inexperienced person could contract it without being aware of what is happening \cite{hinsen2015technical}; In both cases it's often done for saving the limited available resources and shortening time-to-market \cite{tom2012consolidated}. For example, startups are highly pressured to quickly test products and ideas in order to save capitals and be faster than competitors.
%Besker et al. studied how startups incur in TD, which are the factors, challenges and benefits of intentional acquisition of TD; two of the regulating factors found by the authors are the experience of the developers and the software knowledge of the founders \cite{besker2018embracing}.
In contrast to the beneficial viewpoint of TD, Ron Jeffries argues that the metaphor could be `Perhaps too gentle', because it highlights the wise aspect of the choice of contracting a debt; the problem is that people also take debt unwisely. 
Technical debt benefits a software project as long as it is handled before the bigger long term cost is realized.
%
It is generally accepted by the community that technical debt needs to be managed otherwise bad things will happen \cite{guo2016exploring} \cite{hinsen2015technical} \cite{allman2012managing} \cite{martini2018technical}. 

This is why it is important to detect technical debt: avoid additional costs and unwelcome consequences. 
%There are multiple approaches to detect TD:
%The literature shows multiple approaches to automatically detect TD.  

When developers are aware that they introduce technical debt in the source code, for the many reasons we mentioned above, they can decide to leave a code comment to document what they are doing; this is called self-admitted technical debt (SATD) \cite{potdar2014exploratory}. In other words, a SATD is a written confession left as a testimony that there is something `not-quite-right'.
% approaches to detect SATD

%1.1

\section{Objectives and Results}
%Once the context is clear, I will explain the goal of the thesis and summarize the achieved results
There is an extensive literature on techniques to detect technical debt automatically. However, to the best of our knowledge, no study leverages SATD annotations to create a technical debt detector.
We propose a method to use SATD comments to detect technical debt in source code using a deep learning model. 
This thesis describes how to learn a model that classifies code snippets into two categories: TD-free or TD-affected.
To create the dataset needed to learn the model, we processed 245,243 projects commit histories cloned from public GitHub repositories. 
The model internally represents the code snippet as a fixed-length vector. The pre-processing pipeline extracts features from the source code (AST-paths) and they are embedded as code-vectors. Conceptually, this is similar to how word2vec works. The distance between two vectors gives a measure of how similar or dissimilar two snippets are. The last part of the model is a fully connected layer; its purpose is to perform the final binary classification of the code-vector. 

We present a large-scale empirical study to (i) assess the accuracy of our classifier and (ii) research the correlation between the model prediction confidence and the accuracy of the prediction itself. Besides quantitatively evaluating the results, we present a qualitative analysis of concrete high confidence predictions.

% quant results 
We tested the accuracy of the model on two different datasets with unequal sizes. The results using the smaller dataset, shows 71\% for precision, 62\% recall and 66\% for F1-score. The second best precision (67\%) and the second highest F1-score (61\%) comes with the largest dataset.

%quant results with confidence
The results on the confidence level analysis show interesting results: when filtering for a confidence level greater than 0.9 we measured a gain on precision, from 71\% to 78\%; the excluded samples shows their effect also on the recall that goes from 62\% down to 52\%. When using a dataset with bigger code snippets and confidence greater than 0.9 we discard 98\% of the test predictions; however the precision is high as 99\% and the recall drops to 10\%, both due to the (correct and incorrect) discarded predictions.




% The goal of the thesis is to exploit SATD to train a model that can acquire the capacity to distinguish between TD-free code and code affected by TD.
% Using the comments in open source projects I will identify class methods noted as SATD. Through the vcs commit history I will identify when this comments disappear; the assumption here is that when the comment is removed from the code the SATD has been fixed. Many cases are excluded to minimize the probability of keeping a false positive, i.e. the SATD is not fixed but the comment is removed. For example, the simplest case excluded is when the code is exactly the same and the only change is the SATD comment removal. Another reason of exclusion is when the code is changed too much; in such case it would not be prudent to keep the sample in the dataset.

% The achieved results tell us that it is difficult to predict with high accuracy on methods with big bodies. As the train dataset is limited to shorter and shorter method size, the accuracy grows. 


%1.2
\section{Structure of the Thesis}
The remainder of the manuscript is organized as follows.
\begin{itemize}
 \item \textbf{Chapter 2} presents the state of the art in technical debt and self-admitted technical debt. It explores automatic software bugs identification and introduces the basic concepts of code smells, anti-patterns and the prominent literature on these subjects. The last part examines the most recent researches on machine learning applied to technical debt.
 \item \textbf{Chapter 3} explains how we used deep learning to detect technical debt. There are three main sections: dataset creation, deep learning model and hyperparameter tuning.
 \item \textbf{Chapter 4} presents the research questions and details the design and planning of the study.
 \item \textbf{Chapter 5} presents a quantitative analysis of the results that we achieved. We also present qualitative research and related findings.
 \item \textbf{Chapter 6} discusses  threats to validity. We explain which threats we faced and show how we mitigate them.
 \item \textbf{Chapter 7} concludes the work by summarizing our findings and recommends directions for future work.
\end{itemize}




% \\*******
% \\
% It is useful to list some aspects of TD that form the basis for a tractable model of the phenomenon; many of these aspects comes from the already well known attributes of financial debt. The \emph{principal} is the amount that needs to be payed to 'make things right'. The \emph{interest amount} is the added cost in addition to the principal;  

% In order to better analyze and create a tractable model of TD, Alves et al. identifies three variables \cite{martini2018technical}: 


% Identified variables (3)
% Tracking and managing is important
% Few proposed tracking and managing examples

% there is a cost paying the debt and there are interest costs when efforts are wasted coping with non optimal code.



% Why detecting it is important\\
% \\
% the satd is there
% \\
% it’s undocumented
% \\
% the team is not aware of it
% \\
% \\
% Tom et al. in their systematic literature review studied, among others, the benefits and drawbacks of incurring in TD \cite{tom2012consolidated}; the part we are now interested in is the drawbacks: 
% \\
% Increasing costs over time, such as the amount of effort required to deliver a certain amount of functionality 
% \\
% Work estimation becomes difficult
% \\
% Developer productivity is negatively impacted 
% \\
% Becomes increasingly difficult to repay as decisions are affected by existing debt 
% \\
% Increased risk involved in modifications to the system
% \\
% Change becomes prohibitively expensive to the point of bankruptcy, and a complete rewrite
% and new platform may become necessary
% \\
% Decreased quality in the end product
% \\
% \\
% Ci sono team che risolvo as-you-go, altri hanno strategie per rilevare, identificare e fixare.
% It has been studied that TD [Investigating the Impact of Design Debt on Software Quality]
% E' importante rilevare e monitorare i TD perche' questi aumentano il costo di un progetto software, costo dovuto dall'interesse che si deve pagare in relazione al TD.

% \newpage


\chapter{State of the Art}
The following sections describe existing approaches to detect different types of technical debt.

%2.1

%2.1
\section{Automatic Identification of Software Bugs}
This section is about bugs and tools that are capable of detecting them automatically. We divide bugs in three different categories:
\begin{itemize}
    \item Memory related bugs, e.g. null pointer violation, memory leak and buffer overflow.
    \item Concurrency related bugs, e.g. critical race conditions, deadlock and unsafe concurrent data access.
    \item Semantic related bugs, when the fault arise from a contradiction to the intention of the programmer or the original design.
\end{itemize}
\\
Different tools use different methods to detect bugs; the first big distinction to be made is between dynamic and static analysis.

\emph{Static analysis} is performed with no execution of the program and relies only on the source code or on some object code. 

\emph{Dynamic analysis} executes the program and inspect its behaviour at run-time. It can be implemented in many flavours, for example: with instrumentation of the executable, with a virtual processor or taking the form of a scheduler. Two are the main factors that are often taken under observation, one is the performance loss in the execution and the second is the extension of the run-time monitor (both in quality and quantity).
\\
\\
We can define another distinction on how bugs are identified; many tools use rules to detect if some violation has occurred. The rules themselves are of two types: programming and statistical rules.

\emph{Programming rules} are usually clear and squared, e.g. they derive from axioms, mathematical models or are manually defined.

\emph{Statistical rules} are defined through a statistical analysis on multiple samples. Usually there is a training phase where observations are collected and correct rules (invariants) are refined.
\\
\\
Another means of detecting bugs is \emph{model checking} where the tool verify correctness in a usually finite state system. This verification is performed using formal methods on typically formally specified system. 

The last category for bugs finding techniques is the \emph{annotation-based} tools. Those requires the annotation of programs to extract semantics and verify consistency and correctness.
\\
\\
What follows is the description for each selected tool.
\\% \includegraphics{classification-of-bug-detection-tools.png}
\\
\labeltextbf[]{PREfix}{bug:PREfix} \cite{bush2000static} A static analyzer for finding dynamic programming errors

PREFix is a source code analyzer that detects a broad range of errors in C and C++ code.
Its goal is to detect many runtime issues on real world programs, without dynamic analysis and instrumentation; only the source code text is used. PREfix can detect defects efficiently through a model that abstracts functions and their interactions; the analyzer traces execution paths handling multiple language features: pointers, arrays, structs, bit field operations, control flows statements and so on.

The method used by the tool is based on the simulation of individual functions. It employs a virtual machine that simulates the actions of each operator and function call. With the detailed tracking it can report defects information to the user so to easily characterize the detected error.
The tool can be applied both to a complete program source or only a subset. This bottom up approach is particularly useful when the source code is not fully available (e.g. in the case of a thirty part library).
\\
\\
\labeltextbf[]{RacerX}{bug:RacerX} \cite{engler2003racerx} RacerX: effective, static detection of race conditions and deadlocks

RacerX deals with complex multithread systems. It detects race conditions and deadlock using static analysis.
It can infer from the source code the object lock that is assigned to a particular code block. It detects code that is used in a multi-threaded context; it also detects when the code endeavours in dangerous shared access. 

RaceX uses annotations only to mark the code that deals with lock acquisition; this requirement keeps the burden on the user to the minimum to increase ease of adoption. The authors of RaceX report their experiences on the biggest problem about race detection: in large codebase there are massive amounts of unprotected variable access; the key point is to report only those that can actually cause problems. There is emphasis on two aspects:
\begin{itemize}
    \item The first is to minimize the impact of reporting false positive in order to avoid the users to discard the use of the tool; to achieve this, RaceX employs specific techniques to lower the impact of analysis mistakes.
    \item The second is the speed of the tool: the authors keep the time of execution for the analysis under deep scrutiny; they claim that for a codebase of 1.8 LOC the time required is between 2-14 minutes.  
\end{itemize}
\\
The tool has been found capable of finding severe problems in huge projects like Linux, FreeBSD and also in a large closed source commercial software.
\\
\\
\textbf{Purify} \cite{hastings1992fast} Fast Detection of Memory Leaks and Access Errors. Winter 1992 USENIX Conference
 
 Purify is a dynamic analysis tool for software testing and quality assurance . 
 It instruments the object files generated by the compiler (the software dates back to 1992 and the supported platform is Sun Microsystem's SPARC); the process that acts on the object files  include also third-party libraries. 
 Purify detects multiple errors: memory leaks, access error, reading uninitialized memory. The injected instructions check every read and write memory operations; the slow down of the target is under three times in respect to the non-instrumented execution time. 
 Enabling the use of Purify is as simple as adding a word in the makefile; the generated overhead is inside the limit of tolerance of developers and it allows to detect bugs early in the development cycle.
\\
\\
\textbf{Valgrind} \cite{nethercote2007valgrind} Valgrind: a framework for heavyweight dynamic binary instrumentation

Valgrind is a framework available for many Linux, Android and Darwin architecture. 
It is the foundation where many tools are built upon. All these tools, several are already included with the standard distribution, help to build more correct and faster program and are categorized as dynamic analysis tools.\\
The eight supplied tools can be divided in the following groups: memory error detector, cache profiler, thread error detector, heap profiler.
There are also two additional tools that are provided to illustrate how to use the framework works and how to use the core low level infrastructure to implement instrumentation.
\\ 
Basically, the core implements a synthetic CPU that asks the selected tool how to instrument the code and then continues and coordinates the execution. All the instructions are simulated and the memory access is sandboxed; this includes also the third party library linked into the executable. There are ways to manage and suppress every output generated to avoid clutter and unwanted error reports.
\\
It is advised to enable the debug info into the executable; without them Valgrind is unable to determine which function is the owner of a specific instruction, as such it will produce almost useless error and profiling messages. It is also advised to use minimal compiler optimization to avoid incurring into false positive error reports.
\\
The slowdown of the execution depends on the specific instrumentation of the selected tool, and it is roughly between four to fifty times of the original speed. 
%EXPAND% The most popular tool in the suite is Memcheck; 
\\
\\
\labeltextbf[]{CP-Miner}{bug:CP-Miner} \cite{li2006cp} CP-Miner: Finding Copy-Paste and Related Bugs in Large-Scale Software Code

CP-Miner stands for Copy and Pasted code Miner. Zhenmin et al. found that a significant portion of source code in many widely used open source projects are duplication made with copy and paste. This practice introduces bugs, the main reason being that programmers leave identifier untouched instead of renaming them consistently to match the new code context. When the label do not exists in the new place it will be detected with a compilation error; on the other hand, if the identifier exists in the new context it will not be detected by the compiler and it will introduce a hidden bug very hard to detect.\\
CP-Miner tolerates modification to the pasted code. In order to be detected, the segments do not need to be identical; they can also contain insertions or modifications. The tool is capable of detecting duplicated code but not all detection are true positive; nonetheless it is able to report many significant duplicates with hidden issues. The authors detected 28 copy-paste related bugs in the Linux kernel code base and 23 in FreeBSD; these bugs were reported and most of them where previously unknown to the project team.
The analysis is done infra-project and targeted the following large projects: Linux, FreeBSD, PostgreSQL and Apache HTTP server.
\\
The main contributions of the paper are: scalability, bugs detection and statistical study of the copy-pasted code. 
\begin{itemize}
    \item \emph{Scalability} is a strong point of CP-Miner because the technique used in it allow to quickly and efficiently scan large projects including operating system code. For example, it took 20 minutes to find 150,000-190,000 copy-pasted segments respectively in the FreeBSD and Linux kernel; such fragments account for roughly one fifth of the code base. At the time, both projects had more the three million lines of code. 
    \item \emph{Bugs detection} was found to be very effective because of the positive response from the open source project maintainers; most of the reported bugs were not found by static or dynamic analysis detection tools. 
    \item The authors conducted a \emph{statistical study} of the copy-pasted code to give an overview of the phenomenon; the majority of the copied code is between 5 and 16 statements. Around 50 percent of the code has only two copies but around 7 percent has more than eight duplicates. Roughly 12 percent of the copy-pasted code segments are whole functions. Kernel modules are affected with different concentration, depending on the module under analysis: drivers, arch and crypt modules have high copy and paste segments than other parts of the project.
\end{itemize}
\\
\\
\textbf{D. Engler's} \cite{engler2001bugs} Bugs as Deviant Behaviour: A General Approach to Inferring Errors in Systems Code

The approach taken by Engler et al is to extract from source code beliefs and properties that must or could hold. Through static analysis it is automatically detected two different types of beliefs: MUST beliefs and MAY beliefs. The first one is something that must certainly hold, for example the dereference of a pointer holds the credence that the reference is not null and must be valid. The second one is statistical by nature; the code is observed and searched for patterns that suggests beliefs, for example a call to function "x" followed by a call to function "y". The probabilistic nature of the MAY belief comes when validating it: at first it is treated as a MUST belief then a search yields all uses in the source code of such belief (i.e. the use of "x" and "y"). If it turns out that such pattern is respected most of the times then the belief is probably valid, otherwise it is treated as a coincidence and, as such, discarded.
MUST beliefs bear no doubt about their validity and represent internal consistency. All contradictions from both MUST beliefs and valid MAY beliefs are reported as errors (i.e. bugs).
The authors leverage their prior work \cite{engler2000checking} where they used static analysis to fix manual defined rules for specific system; for example a call to spin\_lock(l) must be paired with a following spin\_unlock(l). Such patterns were previously specified by hand; with the current work they enable a system to infer the same (and more) rules automatically. They reported that the automatic system is able to detect all the manual rules plus a considerable additional amount that goes from ten to one hundred patterns more.
The general idea underling this paper, is that the source code contains intrinsic information about what is correct; finding errors in real system means exploiting what is intended as "correct". Sadly, most of the times these rules are not documented, not formalized or if they are available, they are present in informal and unusable format.
With this work the authors use static analysis to extract the beliefs that the programmers infused in the source code, without the need of a priori knowledge.
Manually performing the task of extracting correct behaviours and rules from source code is usually a hard, difficult and daunting experience, particularly in view of multiple releases and big code bases. Engler et al show of being able to apply this techniques to complex systems as Linux and OpenBSD operating systems. The results are hundreds of detected contradictions (i.e. errors or bugs) reported; many of them have been assessed and resulted in kernel patches.
\\
\\
\textbf{DIDUCE} \cite{hangal2002tracking} Tracking Down Software Bugs Using Automatic Anomaly Detection

The paper introduces a tool written by the authors called DIDUCE: Dynamic Invariant Detection $\cup$ Checking Engine.
It is based on instrumentation of Java programs so to observe their behaviour at run-time. During the lifetime of the target Java process, DIDUCE gather information and collects hypothesis of invariants; those are the rules that the program should obey. As violation to the invariants are encountered the tool relaxes the hypothesis allowing for different behaviour. Every invariant has a confidence level, the process previously described updates it; then the user can go through all the anomalies reported ordered by their rank. The tool is intended to be used in the discovery of the root cause of bugs and as an aid in better understanding the program under analysis. 
The query of this ranking can be done, for example, just before a crash occurs: inspecting what DIDUCE detects as anomalies often lead to the discovery of the root cause of the problem.

The paper describes also the findings during the application of DIDUCE to four Java real-life programs, one of which is the JSSE Library (Java Secure Sockets Extension); the bug under analysis was found during the development of a proxy server to be applied to the JSSE Library. The problem was that using the proxy server triggered unexpected behaviour in the JSSE internals. Thanks to DIDUCE the programmer was able to find the issue in the core of the library: it was reported with high confidence that method read() returned a different value than usual. This method returns the number of bytes read from the stream and Java specs clearly define that the method can also return with a buffer not fully filled. A common Java programmers pitfall is to ignore this result and skip the loop on read() to obtain all expected data. DIDUCE reported a violated invariant with high confidence because the result from the method was always 74 and in one instance (just before the Exception) the number was smaller; this information made apparent that using the proxy triggered a bug present in JSSE Library.
The authors' experience suggests that discovering bugs using their proposed methodology is simple across many different kind of programs, shown in the paper with four compelling cases.
%expand the idea of "DIDUCE, what's new? what changed up to a point?"
\\
\\
\textbf{AccMon} \cite{zhou2004accmon} AccMon: Automatically Detecting Memory-related Bugs via Program Counter-based Invariants

The contributions of this paper are two innovative ideas: the first is a novel statistical approach to detect bugs in memory related issues, the second is a novel architectural extension to decrease the overhead of the monitoring process.

The first is called PC-based invariant detection (PC stands for program counter); it leverages the observation that most programs access memory location mostly from the same instructions. Being probabilistic in nature, it can detect memory access anomalies that deviate from the baseline; they usually are the causes for bugs, stack overflows and many other memory-related issues. We can see that there are tho phases: one where the statistical data are gathered and the baseline rules (e.g. invariants) are formed; the other phase put to use the collected rules and check for violations that will be reported.
These two phases are intended to be used in multiple runs but also in the same single long-running execution.


The second contribution is called Check Look-aside Buffer (CLB); it aims to lower the burden of the dynamic process monitor activities to decrease the overhead needed. The authors report their experiments on performance: in the worst case analyzed the loss of speed of the process is less than 3 times. In other tools, this slowdown can be of one order of magnitude greater.

The effectiveness of the authors' contribution are tested through AccMon, a tool they developed in order to implement the idea of a PC-based invariant detection; such experiments shows that the proposed novel ideas are sometime capable of finding more bugs in respect to other tools such as Purify or CCured. Then, in conjunction with previous work from the same authors (iWatcher) it is demonstrated the effectiveness of CLB in lowering the burden of the overhead.

The authors report many other advantages in using AccMon in respect to other tools:
\begin{itemize} 
%\itemsep0em
\item the analysis is not done on the values of the variables, thus It can detect also bugs that do not violate value-based invariants.
\item in the current form, it uses source code to achieve compiler-based optimization but it can directly work with binaries without the need of compilation.
\item it does not need type information; given it's statistical nature it can detect anomalies just using abstract memory pointers.
\item it's possible to switch off the monitoring activities dynamically at runtime, with almost no overhead. The authors states that AccMon can be used in production runs.
\end{itemize}


\noindent \textbf{Liblit's} \cite{liblit2003bug} Bug isolation via remote program sampling

The underling observation that pave the road for this paper is that often the user community of a program has more raw throughput of running and executing it than the developers. In other words, the number of executions that the team responsible for the program can apply for testing is dwarfed by the number of executions that the community can or will bring up to bear.
The authors propose an infrastructure for gathering data from the user's execution to a central information store; then they propose a process, called automatic bug isolation, to analyze gathered data in order to provide information to the developers to help find and fix bugs.
This infrastructure shows multiple benefits:

\begin{itemize} 
\item Use a vast amount of data that is generated by the execution of the program by the user community; it is usually discarded and do not contribute to better the quality and the experience for the user itself.
\item Enabling the collection of information helps to draw a clearer picture on the effective use of the program and drive better decisions about development roadmap
\item Map and define feature usage statistics
\item Avoid issues related to manual feedback generated from an user intervention: usually the user is unsophisticated and non technical; it's ability to have a positive impact on the bug reporting is limited. The benefits of automatically gather this information are many and varied.
\end{itemize}

Designing an infrastructure that was able to scale was a non-trivial process; there are two main issues to address.

One is to make the lowest impact on the performance on the program execution. It is very important to be respectful of the resources used in gathering debug information. To achieve this goal, the authors employ sampling; they also address a technique to conduct fair sampling.

The other one is a craftiness in gathering and periodically sending data to the central system: even collecting a small amount of information have an huge impact on scalability.

The authors then focus on the data analysis phase. They propose three different application with increasing level of sophistication:

\begin{itemize}
    \item They show how to share the burden of assertion across the use base so to inflict upon each user only a small fraction of the checks.
    \item They show how to start from a large set of predicates (predicate guessing) and shrinking it down over time to reveal the smallest set that can deterministically predict a bug.
    \item They show how to use linear regression to isolate non-deterministc bugs; in other words they shrink the set of predicates that has the highest correlation with the failure.
\end{itemize}
\\
\\
\textbf{ESC/Java} \cite{flanagan2002extended} Extended Static Checking for Java

This tool perform static analysis on Java source code looking for errors and warnings to report. It provides specific Java annotation to formally express design decisions. 
ESC handles and warns about multiple common programming errors, e.g. null dereference, index out of bounds, types errors; it also warns about concurrency errors, like race conditions and deadlock.
Aided by the custom annotations, ESC employs an engine to decode the semantics of the program and apply techniques to automatically prove theorems; doing so it is able to report potential bugs that are not detected by the type system and that are detected at runtime.


One core requirement imposed on ESC is the modularity of checking; in other words, it can work on pieces of code (i.e. methods) in isolation to the rest of the code. This restriction was chosen for scalability reasons even if the downside is the need of custom annotations.
The authors argue that the cost of using the annotations are not an hard overhead: when developers are engaged in manual code review they need information that usually come from unstructured sources (e.g. natural language comments) and they already sustains the burden of gathering additional knowledge not present in the raw code.

It is evident throughout all the paper the importance the authors put on the tradeoff between the cost of the annnotation process and the benefit of true errors feedback; this sentence taken from the paper's introduction sums the core of this tradeoff: "if the checker finds enough errors to repay the cost of running it and studying its output, then the checker will be cost-effective, and a success".
Infact, they enumerates two important features needed by an ideal static checker : 
\begin{itemize}
    \item soundness - if the program has errors, it will find some
    \item completeness - every error reported is a true positive
\end{itemize}
ESC do not seek to honor these two features for the very belief quoted above. The authors observe that the alternative processes used to achieve software quality (testing and code reviews) do not possess any of the features of the ideal static checker.

At paper time of writing, ESC was used for two years on multiple kind of programs and was proven effective in finding meaningful bugs. The performance was adequate for interactive use on most methods. 
Even if it was proven of real usefulness, the users' feedback suggest that the cost of annotating was high and the number of warnings was excessive.
It must be said that the annotation were added after the development of the project and not during the evolution of it; this is commonly know by developers to be a dreaded task. 
\\
At this point it is unclear if the tool delivers a positive tradeoff between cost and benefit; the experience of the authors during internal use of the tool was encouraging. They believe that ESC is already a valid tool to be used in classrooms: it enforces good design, modularity and verification.
\\
\\
\textbf{VeriSoft} \cite{godefroid1998model} Model checking without a model: An analysis of the heart-beat monitor of a telephone switch using VeriSoft

The goal of VeriSoft is to detect problems in a concurrent reactive system (CRS) through the exploration of its state space. The \emph{reactive} word stands to describe the continuous interaction of the system with the environment.
The issues detected are: deadlocks, assert violations, livelocks. 
A CRS is composed of two parts: a finite number of processes written in arbitrary code (e.g. C, C++, Java, tcl, and so on) and a finite number of communication objects (TCP connections, semaphores, shared memory, and so on).
The need of such tool arise from the difficulty of writing a robust and reliable CRS; it is well accepted that concurrent systems are prone to unexpected issues, difficult to track, test and to reproduce.

What VeriSoft does is a systematic state space exploration; it defines the state space as a directed graph: the nodes are the global states and the edges are the transitions between states. 
It follows that each global state should be uniquely identified and this is one of the core issue that VeriSoft solves with an original combination of algorithms; the author calls it "an efficient state-less search".

The paper reports the analysis conducted with VeriSoft on a software owned by Lucent Technologies: "Heart-Beat Monitor" (HBM). Such software monitors the status of a telephone switch elements and determines their state based on the propagation delays of messages sent through those elements.
HBM is an important piece of software because it has a big impact on the switch performance due to its influence on the switch routing.

The experience of the authors is reported as successful: they were able to find errors in the documentation and in the software itself. Subsequently they modified the code to strengthen some properties and tested them again with VeriSoft; after another run, as desired, VeriSoft reported satisfactory results.
The development team of HBM decided to integrate the code changes from the authors for the next commercial release.

VeriSoft acts as a scheduler and has complete control over non-determinism so it can reproduce any interesting scenario (i.e. those that during the automatic tests led to errors and issues).
One other benefit is that there is no need to describe the model with specific languages: it relies on the exercise of the actual code.
One downside is that it cannot detect cycles in the graph of the global spaces and, as such, it can only detect violation of safety properties.
\\
\\
\textbf{JPFinder} \cite{havelund1999applying} Applying Model Checking in Java Verification

Java PathFinder (Jpf) is a prototype translator from the Java language to Promela (Process Meta Language). Promela runs on SPIN (Simple Promela Interpreter).
SPIN is a general tool to find concurrency problems and verify the correctness of a system.

It is reported that Jpf is not the first attempt of Java-to-Promela translator; in addition to this other work, Jpf can handle a significant number of features of the Java language. At the same time, many other are missing.
The paper describe the issues due the impedance between Java and Promela; they come in two flavors: performance issues and missing feature issues.

Jps provides the programmer with Java static methods to annotate the source code with assertions; those assertions will be checked with the SPIN model checker.

The authors used a Chess game server written in Java as the test subject for finding synchronization bugs. They did not use the original source code but wrote a simplified abstraction in Java composed with 16 classes and roughly 1400 lines of code; it is reported that this Java program was non trivial and the development was done without thinking about formal verification.
Then the authors fed the Java semplification to Jpf and were able to find a bug that was later confirmed. 
\\
\\
\textbf{CMC} \cite{musuvathi2002cmc} CMC: A pragmatic approach to model checking real code

Model checking is very hard in practice; it usually involves the use of a specific domain language to describe the model and then a model checker. This common approach to model checking is very hard to endure in practice: it exposes the age-old dualism of having two parallel systems.
On one hand we have the actual implementation, on the other hand we have the abstraction that represents the model of the implementation.
Having two distinct bodies opens the following issues:
\begin{itemize}
    \item The model could exhibit issues that are not present in the actual implementation.
    \item The implementation could show bugs that are not present or detected in the model.
    \item The need to maintain both systems coping with the impedance of two different ways of expressing meaning, behaviour and intent.
\end{itemize}


Complex systems often hides rare but nasty bugs that arise only after many weeks of continuous run; this is a major issue with such systems.
Explicit checkers can help in this scenario; they search a huge state space without wasting the resources for repeated parts of usual testing.
\\
\\
The first contribution reported by the authors is CMC: C and C++ model checker.
It works directly on the implementation without the need to create a separate model to be checked. CMC needs some adaptations on the code, some are just good programming practices (e.g. asserts, specifying the environment), other are changes required specifically by CMC: one is for handling the non-determinism and another one is to handle the initialization functions and event handlers.
The tool works by directly executing and scheduling the system under analysis. It needs to store and load the whole state space and as such it handles techniques to cope with the \textit{state explosion problem}: simple heuristics help to prune a huge amount of states.
\\
\\
The second contribution is the application of CMC to three implementations of AODV routing protocol. The actual goal of CMC is to check network code implementations, but the ultimate goal is to check a broader range of programs.
During the experimentation the authors, through CMC, were able to find 34 errors many of whom were meaningful errors. Actually, their work exposed also a bug with the AODV specification itself (that was later acknowledged in the RFC 3561 citing the first author of this paper).

Network code is of core importance for the stability of a system; it is prone to many issues that undermine correctness, e.g. packet loss, hardware errors, security attacks. 
\\
\\
CMC proved to work well, given the results on three different implementation of a routing protocol.
For a wider acquisition of CMC it essential for the authors, to lessen the burden of the code adaptation and automate it as much as possible.



%2.2
\section{Code Smells and Antipatterns}
\textbf{copy\&paste}
Several techniques have been proposed in the literature to detect code smell instances affecting code components, and all of these take their cue from the suggestions provided by four well-known books: [29], [16], [86], [66].

\subsection{Four inspirational books description}
\textbf{copy\&paste}
\\
%smell-book-a
\textbf{Webster [86] - todo}
The first one, by Webster \cite{webster1995pitfalls} defines common pitfalls in Object Oriented Development, going from the project management down to the implementation.
\\
\\
%smell-book-b
\textbf{Riel [66] - todo}
Riel [66] describes more than 60 guidelines to rate the integrity of a software design
\\
\\
%smell-book-c
\textbf{Fowler [29] - todo}
The third one, by Fowler [29], describes 22 code smells describing for each of them the refactoring actions to take. 
\\
\\
%smell-book-d
\textbf{Brown et al. [16] - todo}
Brown et al. [16] define 40 code antipatterns of different nature (i.e., architectural, managerial, and in source code), together with heuristics to detect them.
\\
\\
\subsection{Last decade proposed approaches}
From these starting points, in the last decade several approaches have been proposed to detect design flaws in source code - todo fix copy\&paste
\\
\\
%smell-1
\textbf{Travassos et al. \cite{travassos1999detecting} }
Travassos et al. created a set of techniques for manually identifying defects in order to improve software quality. These practices help individuals to read object oriented code and assess, through a predefined taxonomy, if some problem is present. The authors conducted an empirical study on these techniques and report their feasibility.
\\
\\
%smell-2
\textbf{van Emden and Moonen [83] todo, fix copy\&paste }
van Emden and Moonen [83] presented jCOSMO, a code smell browser that visualizes the detected smells in the source code. In particular, they focus their attention on two Java programming smells, known as instanceof and typecast. The first occurs when there are too many instanceof operators in the same block of code that make the source code difficult to read and understand. The typecast smell appears instead when an object is explicitly converted from one class type into another, possibly performing illegal casting which results in a runtime error.
\\
\\
%smell-3
\textbf{Simon et al. [] todo, fix copy\&paste }
Simon et al. [72] provided a metric-based visualization tool able to discover design defects repre- senting refactoring opportunities. For example, a Blob is detected if different sets of cohesive attributes and methods are present inside a class. In other words, a Blob is identified when there is the possibility to ap- ply Extract Class refactoring.
\\
\\
%smell-4
\textbf{Marinescu [50] todo, fix copy\&paste }
Marinescu [50] proposed a metric-based mechanism to capture deviations from good design principles and heuristics, called “detection strategies”. Such strategies are based on the identifica- tion of symptoms characterizing a particular smell and metrics for measuring such symptoms. Then, thresholds on these metrics are defined in order to define the rules
\\
\\
%smell-5
\textbf{Lanza and Marinescu  [44] todo, fix copy\&paste }
Lanza and Marinescu [44] showed how to exploit quality metrics to identify “disharmony patterns” in code by defining a set of thresholds based on the measurement of the exploited metrics in real software systems. Their de- tection strategies are formulated in four steps. In the first step, the symptoms characterizing a smell are defined. In the second step, a proper set of metrics measuring these symptoms is identified. Having this information, the next step is to define thresholds to classify the class as affected (or not) by the defined symptoms. Finally, AND/OR operators are used to correlate the symptoms, leading to the final rules for detecting the smells.
\\
\\
%smell-6
\textbf{Munro [53] todo, fix copy\&paste }
Munro [53] presented a metric-based detection tech- nique able to identify instances of two smells, i.e., Lazy Class and Temporary Field, in the source code. A set of thresholds is applied to some structural metrics able to capture those smells. In the case of Lazy Class, the metrics used for the identification are Number of Meth- ods (NOM), LOC, Weighted Methods per Class (WMC), and Coupling Between Objects (CBO).
\\
\\
%smell-7
\textbf{Moha et al. [51] todo, fix copy\&paste }
Moha et al. [51] introduced DECOR, a technique for specifying and de- tecting code and design smells. DECOR uses a Domain- Specific Language (DSL) for specifying smells using high-level abstractions. Four design smells are identified by DECOR, namely Blob, Swiss Army Knife, Functional De- composition, and Spaghetti Code.
\\
\\
%smell-8
\textbf{Tsantalis and Chatzigeorgiou [79] todo, fix copy\&paste }
Tsantalis and Chatzigeorgiou [79] presented JDeodor- ant, a tool able to detect instances of Feature Envy smells with the aim of suggesting move method refactoring op- portunities. For each method of the system, JDeodorant forms a set of candidate target classes where a method should be moved. This set is obtained by examining the entities (i.e., attributes and methods) that a method accesses from the other classes. In its current version JDeodorant8 is also able to detect other three code smells (i.e., State Checking, Long Method, and God Classes), as well as opportunities for refactoring code clones.
\\
\\
%smell-9
\textbf{Ligu et al. [47] todo, fix copy\&paste }
Ligu et al. [47] introduced the identification of Refused Bequest code smell using a combination of static source code analysis and dynamic unit test execution. Their approach aims at discovering classes that really want to support the interface of the superclass [29]. In order to understand what are the methods really invoked on subclass instances, they intentionally override these methods introducing an error in the new implementation (e.g., division by zero). If there are classes in the system invoking the method, then a failure will occur. Otherwise, the method is never invoked and an instance of Refused Bequest is found.
\\
\\
\subsection{Code smell detection formulated as an optimization problem:}
%smell-10
\textbf{Kessentini et al. [36] todo, fix copy\&paste }
Kessentini et al. [36] as they presented a technique to detect design de- fects by following the assumption that what significantly diverges from good design practices is likely to represent a design problem. The advantage of their approach is that it does not look for specific code smells (as most ap- proaches) but for design problems in general. Also, in the reported evaluation, the approach was able to achieve a 95\% precision in identifying design defects [36].
\\
\\
%smell-11
\textbf{Kessentini et al. [37] todo, fix copy\&paste }
Kessentini et al. [37] also presented a cooperative parallel search- based approach for identifying code smells instances with an accuracy higher than 85\%.
\\
\\
%smell-12
\textbf{Boussaa et al. [13] todo, fix copy\&paste }
Boussaa et al. [13] proposed the use of competitive coevolutionary search to code-smell detection problem. In their approach two populations evolve simultaneously: the first generates detection rules with the aim of detecting the highest possible proportion of code smells, whereas the second population generates smells that are currently not detected by the rules of the other population.
\\
\\
%smell-13
\textbf{Sahin et al. [68] todo, fix copy\&paste }
Sahin et al. [68] proposed an approach able to generate code smell detection rules using a bi-level optimization problem, in which the first level of optimization task creates a set of detection rules that maximizes the coverage of code smell examples and artificial code smells generated by the second level. The lower level is instead responsible to maximize the number of code smells artificially generated. The empirical evaluation shows that this approach achieves an average of more than 85\% in terms of precision and recall.
\\
\\
%different category
\subsection{Non binary classification (smell/clean)}
The approaches described above classify classes strictly as being clean or anti-patterns, while an accurate analysis for the borderline classes is missing [40] 
\\
\\
%smell-14
\textbf{Khomh et al. [40] todo, fix copy\&paste }
Khomh et al. [40] proposed an approach based on Bayesian belief networks providing a likelihood that a code component is affected by a smell, instead of a boolean value as done by the previous techniques.
\\
\\
%smell-15
\textbf{Oliveto et al. [58] todo, fix copy\&paste }
This is also one of the main characteristics of the approach based on the quality metrics and B-splines proposed by Oliveto et al. [58] for identifying instances of Blobs in source code
\\
\\
%different category
\subsection{Usage of historical data for code smells}
%smell-16
\textbf{Ratiu et al. [65] todo, fix copy\&paste }
Ratiu et al. [65] proposed to use the historical information of the suspected flawed structure to increase the accuracy of the automatic problem detection.
\\
\\
%smell-17
\textbf{Palomba et al. [60] todo, fix copy\&paste }
Palomba et al. [60] provided evidence that historical data can be successfully exploited to identify not only smells that are intrinsically characterized by their evolution across the program history – such as Divergent Change, Parallel Inheritance, and Shotgun Surgery – but also smells such as Blob and Feature Envy [60].
\\
\\
%2.3
\section{Self-Admitted Technical Debt} % proposal for a new title "Technical Debt and Self-Admitted Technical Debt

Throghout all this section Technical debt will (could) be shortened to TD and foo bar satd
\\
\\
%satd-9
\textbf{Potdar and Shihab} \cite{potdar2014exploratory} An exploratory study on self-admitted technical debt

The authors conducted an empirical study on four open source projects, focusing on three main research questions; they aimed to:
\begin{itemize}
    \item RQ1 finding the concentration of SATD in the projects
    \item RQ2 discovering the reasons for introducing the SATD
    \item RQ3 calculating the per of SATD removal after its introduction
\end{itemize}

They found that 2.4\% to 31\% of the files contained SATD. An interesting finding is that the experienced practitioners are the most likely to introduce SATD. On the other hand, a counter-intuitive discovery is that the amount of SATD has correlation with neither complexity nor time pressure. The removal ratio was found to be roughly between 0.26 and 0.63.
%add more, e.g. 62 satd recognized
\\
\\
%satd-1
\textbf{Storey et al.} \cite{storey2008todo} TODO or to bug: Exploring How Task Annotations Play a Role in the Work Practices of Software Developers

This empirical study has the goal to shed light on how the developers behave on personal and team tasks, in respect to source code annotations (i.e. comments).
The authors analyze the relations that annotations have with common used tools like, e.g. wikis, issue and bug trackers. In order to do so, they gathered and combined data coming from a mix of methods, divided in phases:
\begin{itemize}
    \item Phase 1. It was conducted a survey targeting users of Eclipse IDE. The topic was about annotations: if they did write them, which types and the use of them.
    \item Phase 2. The authors did contextual interviews with developers on three open source projects. Then augmented the answer from the interview with direct analysis on many versions of the source code related to the annotation in question.
\end{itemize}

In the conclusion is reported how these finding could be useful to improve tooling and software process.
\\
\\
%satd-2
\textbf{Guo et al.} \cite{guo2011tracking} Tracking technical debt - An exploratory case study

This paper aims to highlight and make evident the effects of technical debt on the cost and management of a software project. Through the tracking of a single delayed task in a real project, the authors analyze the effect of such technical debt. They created a framework for the explicit management of TD and then applied it, with a simulation, to the real scenario under scrutiny.
The objective of this study is:
\begin{itemize}
    \item determine technical debt effects on the project and evaluate their impact
    \item after the application of the simulation, determine if the framework provided real gain and uncover benefits.
\end{itemize}
The results of this simulation made clear that careful planning and analysis of TD is of high importance: in retrospect the cost of the delayed task almost tripled the cost for the project.
\\
\\
%satd-3
\textbf{Klinger et al.} \cite{klinger2011enterprise} An enterprise perspective on technical debt

This short but meaningful study see the design of an interview to four IBM technical architect. One of the idea was to broad the view about TD from the perspective of the single developer to the perspective of an enterprise.
Starting from the premise that TD can be leveraged as a financial asset (i.e. incur in TD today to gain competitive advantage and repay tomorrow) the study and the interviews are structured to assess how an enterprise handle TD from these standpoints:
\begin{itemize}
    \item how decisions to acquire TD are conducted
    \item what is the leverage earned contracting TD
\end{itemize}
The following are some of the findings:
\begin{itemize}
    \item two different sources of unintentional contraction of TD: from non-technical stakeholders (e.g. fixing a stringent release date at the expense of software quality) and from external forces (e.g. changes in the market and acquisitions).
    \item the process of acquiring TD was informal. The decision had no written records or written analysis on the impact, effects and expectations of such choice
    \item scarcity of knowledge and awareness on the consequences of taking on TD, insufficient channels of communication and lack of a common vocabulary to express contracted costs 
\end{itemize}
\\

\\
\\
%satd-4
\textbf{Kruchten et al.} \cite{kruchten2012technical} Technical debt: From metaphor to theory and practice

This article expands the original metaphor of technical debt by Cunningham in search of a better definition that enables reasoning on a variety of technical debt.
The authors want to lay a theoretical foundation to help the challenge of dealing with TD. These are the main points covered by this work:
\begin{itemize}
    \item TD Landscape. It's a possible organization of the many aspects of software improvement. It divides between visible elements (e.g. new features and defects) and mostly invisible (e.g. architecture and code). The idea is that TD is limited to the invisible part
    \item Tackling of TD. The authors reason about the root causes of TD (e.g. carelessness, lack of education and poor processes) and describe which steps can effectively handle TD (e.g. awareness, explicit management, understand what tools can and cannot do, nurture architecture, documentation)
    \item Unified theory. It is observed that the challenge is making the right sequence of changes to improve the software; in respect to this, perhaps the financial and economic models could be the underling layer to the TD landscape (i.e. express all the changes tied to their cost and value over time).
\end{itemize}
\\
\\
%satd-5
\textbf{Lim et al.} \cite{lim2012balancing} Technical debt: towards a crisper definition report on the 4th international workshop on managing technical debt
 
Lim et al. conducted an interview study on 35 practitioners aimed to define the perceived characteristics of technical debt and in what context TD was encountered. What emerged is most of the teams know well TD and it is an unavoidable necessity in the business reality. Because its certainty one key factor is active management of it: recognition, tracking, analysis, cogent decisions and prevention of worst consequences.
 
The participants were queried with both specific and open questions. Aside of general demographic questions, they were asked to describe an example of TD with its properties, causes, effects and benefits. 
The answer pointed to a different root cause than sloppy programming, poor discipline. Most of the testimony acknowledged that TD was acquired through an intentional decision; some of those decisions were the results of short-term thinking, yielding to the pressure of the moment.
The negative effects of TD were perceived as long term (e.g. the fear to change code expecting to break other parts of the system). 
In some cases it was clear that the benefits were far repaid, in others it was not clear if the balance was positive.
The respondents provided many examples of situations that provided the crucible for TD (e.g. contracts with string deadline, exploiting market opportunity windows). 

The interviewees reported some of their strategies to handle TD:
\begin{itemize}
    \item do nothing. In those parts where low maintenance is required, it's safer to leave things as they are
    \item establish a policy to use development resource to fix TD (5 to 10 percent on total resources)
    \item communication and open dialog about TD between all parties involved (technical and non-technical stakeholders and customers)
    \item make TD explicit and visible to all the developers (e.g. through audits) and keep track of the discoveries.
\end{itemize}
%satd-6 - SKIP it's a workshop 
% \textbf{Kruchten et al. [24]}
% Kruchten et al. [24] reported their understanding of the technical debt in industry as the result of a four year interaction with practitioners.
% \\
% \\
%satd-7
\textbf{Zazworka et al.} \cite{zazworka2013case} A case study on effectively identifying technical debt
% high valuable information left over that can potentially be added
This paper conducted a study to compare manual and automatic technical debt detection. The manual detection was implemented through a questionnaire undertaken by five developers in the same team. The automatic detection was performed using three stable and established tools. 
All participants reported different debt (except in on case) so there is almost no consensus in the human component, on the other hand, the results show a good overlap between manual and automatic detection regarding defect debt. Human intervention is still needed for the other types of debt: documentation, design, testing and usability debt; they were, for the most part, unrecognized by automatic tool.
\\
\\
%satd-8
\textbf{Spinola et al.} \cite{spinola2013investigating} Investigating technical debt folklore: Shedding some light on technical debt opinion

The goal of this paper is to provide some guidance on new research questions about TD. Exploiting the folklore extracted from grey literature, the authors gather 14 statements; then they proceeded to survey to 37 practitioner asking their level of agreement/disagreement. The most agreed upon statement was "If technical debt is not managed effectively, maintenance costs will increase at a rate that will eventually outrun the value it delivers to customers".

The underling observation of this paper is that common belief, traditional stories and customs (i.e. folklore) can help the discovery of interesting topics; then, the agreement of knowledgeable people on those concepts could give a measure of value and worthiness and guide possible future research.
\\
\\

%satd-10
\textbf{Alves et al.} \cite{alves2014towards} Towards an ontology of terms on technical debt

Alves et al. proposed an ontology of terms on technical debt. They developed a \emph{lightweight domain ontology}, designed the quality criteria, conducted a systematic literature mapping and finally submitted the result to a specialist for an evaluation.

The main contribution of this work is to gather information that was spread out and organize a common vocabulary for the field of TD. This common ground wants to help researchers and practitioners evolving the Technical Debt Landscape \cite{izurieta2012organizing}.
The first contribution is the organization of 13 types of TD: architecture, build, code, defect, design, documentation, infrastructure, people, process, requirement, service, test automation and test debt.
The second contribution consists in the organization of indicators themselves; these indicators were used to support identification of the TD.
\\
\\
%satd-11
\textbf{Maldonado and Shihab} \cite{maldonado2015detecting} Detecting and quantifying different types of self-admitted technical debt

The contribution of this paper is the classification of SATD types in four open source projects. The authors manually classified 33093 comments; these are the findings with the range of presence across projects: design debt (42\% - 84\%), requirement debt (5\% - 45\%), defect debt (4\% - 9\%), test debt (0\% - 7\%) and finally documentation debt (0\% - 5\%).

The projects were chosen in the Java realm in different domains with well commented sources: Apache Ant, Apache JMeter, ArgoUml, Columba and JFreeChart. 
Using JDeodorant as comments extractor, the authors gathered more than 166K comments. This number decreased to roughly 33K thanks to processing and filtering of those comment with low likelihood of being SATD. Such operation was conducted through four simple heuristics that targeted the following cases: license comments (removal), commented source code (removal), javadoc (removal), multi line comments instead of block comments (joining). 
The classification process made evident that one SATD can belong to multiple categories (e.g. a design debt can also be a defect debt at the same time). For the sake of clarity this paper will associate only one class to the SATD; in case of ambiguity between multiple types, the more meaningful one was chosen. 
The set of possible SATD classes is taken from Alves et al. \cite{alves2014towards} paper. It is observed that not all 13 original TD classes are found in the selected open source projects; Maldonado and Shihab argues that some of the technical debt are not likely to be reported in written comments (e.g. people and infrastructure debt).
The authors reports that the personal bias and subjectivity can be threats to internal validity (the manual classification was executed by the first author). Other factors on internal validity: quantity and quality of comments could be affected by biased filtering.
About external validity, the authors consider the domain of the projects: it is diverse but all of them are open source Java projects; thus,  the results may not generalize to other languages or market segments.
% This work is closely related to our RQ1, where we man- ually classified the type of technical debt reported in a statistically significant sample of 366 comments. The main differences between our study and the work by Maldonado and Shihab are the scale of the study and the procedure adopted to classify the technical debt instances. Maldon- ado and Shihab analysed a much larger set of comments reporting SATD (33K against 366); However, only one of the authors classified the debt types on the basis of his personal opinion. In our study, we adopted an open coding procedure performed by the two authors to classify the analysed instances and reduce the subjectivity bias. Also, we further refined the technical debt categories defined by Alves et al. [2] as a result of the open coding procedure.
\\
\\
%satd-12
\textbf{Wehaibi et al.} \cite{wehaibi2016examining} Examining the impact of self-admitted technical debt on software quality

This empirical study on fie open source projects wants to explore the relation between self-admitted technical debt and defects in source code. The results reported is that there is an increase of defects after the introduction of SATD. It's also clear that introducing a SATD makes much harder the changes to the related code.
\\
\\
%2.4 INSERTED
\section{TD and machine learning}

--this section will reference the paper from CSABA (techdebt conference 2020 related paper)--
\\
(c\&p) Several detection techniques and tools have been proposed in the literature. Recently, the adoption of machine learning techniques to detect bad smells became a trend [20]
\\
\\
%ML-1
\textbf{Khom et al. [27] and [28]}
Khomh et al. proposed a Bayesian approach which initially converts existing detection rules to a probabilistic model to perform the predictions [27].
Khom et al. [28] extend [27] by the introduction of Bayesian Belief Networks, improving the accuracy of the detection.
\\
\\
%ML-2
\textbf{Maiga et al.}
Maiga et al. proposed an SVM-based approach that uses the feedback information provided by practitioners [35, 36]
\\
\\
%ML-3
\textbf{Amorim et al. [3]}
Amorim et al. [3] presented an experience report on the effectiveness of Decision Trees for detecting bad smells. They choose these classifiers due to their interpretability [3]. Thus, most of the proposed works focus on only one classifier. They were also trained in a dataset composed of few systems and, consequently, the results may be positive towards their aproach due to overfitting.
\\
\\
%ML-4
\textbf{Fontana et al. [21]}
Fontana et al. evaluated different machine learning algorithms on a set of different systems [21]. Their work was later extended and refined, providing a larger comparison of classifiers [20]. The notorious impact of this work was the incredible performance re- ported. Even naive algorithms were able of achieving great results using a small training dataset. This draws attention to possible drawbacks and limitations of their work, which was later reported by Di Nucci et al. [15]
\\
\\
%ML-5
\textbf{Di Nucci et al. [15]}
Di Nucci et al. [15]. They replicated the study and verified that the reported performance was highly biased by the dataset and the procedures adopted, such as unrealistic balanced dataset, in which one third of the instances were smelly.
\\
\\
%ML-6
\textbf{Daniel Cruz at al. --the paper itself-}
Detecting Bad Smells with Machine Learning Algorithms: an Empirical Study.
Bad smells are symptoms of bad design choices implemented on the source code. They are one of the key indicators of technical debts, specifically, design debt. To manage this kind of debt, it is important to be aware of bad smells and refactor them whenever possible. Therefore, several bad smell detection tools and techniques have been proposed over the years. These tools and techniques present different strategies to perform detections. More recently, machine learning algorithms have also been proposed to support bad smell detection. However, we lack empirical evidence on the accuracy and efficiency of these machine learning based techniques. In this paper, we present an evaluation of seven different machine learning algorithms on the task of detecting four types of bad smells. We also provide an analysis of the impact of software metrics for bad smell detection using a unified approach for interpreting the models’ decisions. We found that with the right optimization, machine learning algorithms can achieve good performance (F1 score) for two bad smells: God Class (0.86) and Refused Parent Bequest (0.67). We also uncovered which metrics play fundamental roles for detecting each bad smell.
\\
\\


%2.4
\section{Summing Up (WAS 2.4)}
Here I explain why what I did is different
\section{Summing Up}

There exist many approaches to detect self-admitted technical debt and technical debt in general. However, to the best of our knowledge, there is no study that leverages the SATD annotations to learn a model that automatically detects technical debt using no meta-information except the bare source code.
In this work we create a dataset of method body pairs (affected with SATD and the fixed counterpart) and train a deep learning model that detects the presence or absence of technical debt in an unseen source code snippet. The study aims to investigate how much the model is able to learn and the correlation of the confidence level on the prediction outcome. Then we analyze where and why it succeeds or fail the prediction, in the context of a high level confidence.




\chapter{Using Deep Learning to Detect Technical Debt}

Here I describe the approach.  I'm going to start from a short overview, then I go into details



Using SATD in order to detect technical debt.
There is knowledge inside comments. We want to use this knowledge to train a deep neural network so it can learn to detect technical debt. 
In order to do so, we mine repositories/commits/comments, we find SATD, we detect when they disappear (the assumption is that when they disappear they are fixed) so we have a pre-image and a post-image. pre image is code with satd, post image is fixed.
This construction yields a balanced dataset. We have 50\% of the sample classified as good code, the other 50\% is code affected with technical debt.

SATD are source comments where a programmer confesses that there is something `not-quite-right' 


%3.1
\section{Mining SATD Instances and their Fixes}


%3.2 
\section{The Deep Learning Model}

%3.3
\section{Hyperparameter Tuning}

\chapter{Empirical Study Design}

%Start with a short description about the goal/research questions
%select dbsatds.parent_count,dbsatds.valid , dbsatds.accept, count(*) from dbsatds group by 1,2,3
% 93,061: parent_count, valid, accept = 1

%
% select count(distinct url) from dbsatds  where  dbsatds.parent_count = 1  AND dbsatds.valid = 1 AND dbsatds.accept = 1
%19,395


%select success,done, count(*) from dbrepos group by 1 ,2
%success|done|count |
%-------|----|------|
%      0|   1|  3629|
%      1|   1|245243|
%tot 248'872

The goal of this study is to evaluate the accuracy of our approach for technical debt identification.
The context of the study consists of 93,061 eligible method bodies annotated with SATD and their 93,061 fixed counterparts. These were harvested from 245,243 public Java GitHub repositories. 
% for context selection: The total number of repositories scanned is 245,243

In this thesis, we answer the following research question:
What is the accuracy of our approach identifying methods affected by technical debt?


%4.1
\section{Context Selection}
% Context Selection [explain the dataset used for the evaluation]

We used GitHub GraphQL API \footnote{https://docs.github.com/graphql} to retrieve 7,265,342 urls of public Java repositories created between the start of year 2000 and the end of 2019, thus spanning a time window of 20 years.
The GraphQL api allowed us to request two additional information for each repository: the number of issues and the number of commits. To exclude smaller and less meaningful repositories, we kept only those urls with issue or commit count greater than 100. This reduced the number of repository urls to 248,872.

After the clone phase we were left with 245,243 repositories because 3,629 were rejected or failed:
\begin{itemize}
    \item 1,726 Android OS repository clones were rejected.
    \item 587 clones failed because they were removed or made private.
    \item 1,316 for other mixed reasons.
\end{itemize}



% The dataset is split as follows:
% \begin{itemize}
%     \item 75\% training dataset
%     \item 15\% validation dataset
%     \item 15\% test dataset
% \end{itemize}


%4.2 
\section{Data Collection and Analysis}
%Data Collection and Analysis [How you compute the results]

% select count(*) from dbsatds = 141400

% select dbsatds.parent_count,dbsatds.valid , dbsatds.accept, count(*) from dbsatds group by 1,2,3 =
% parent_count|valid|accept|count|
% ------------|-----|------|-----|
%           1|    1|     1|93061|
%           2|    1|     1|26223|
%           1|    0|     1|11762|
%           2|    0|     1| 4015|
%           1|    0|     0| 3802|
%           2|    0|     0|  844|
%           1|    1|    -1|  423|
%           1|    1|    -2|  373|
%           1|    0|    -1|  225|
%           2|    1|    -1|  175|
%           2|    1|    -2|  115|
%           3|    1|     1|   83|
%           1|    1|     0|   82|
%           2|    0|    -1|   68|
%           1|    0|    -2|   39|
%           3|    0|     1|   22|
%           2|    1|     0|   17|
%           5|    1|     1|   15|
%           4|    1|     1|   14|
%           5|    0|     1|    7|
%           2|    0|    -2|    5|
%           4|    0|     1|    5|
%           6|    1|     1|    3|
%           21|    0|     1|    3|
%           7|    0|     1|    2|
%           8|    1|     1|    2|
%           10|    1|     1|    2|
%           12|    1|     1|    2|
%           21|    1|     1|    2|
%           3|    0|    -2|    1|
%           3|    0|    -1|    1|
%           3|    1|    -1|    1|
%           6|    0|     0|    1|
%           6|    0|     1|    1|
%           8|    0|     1|    1|
%           10|    0|     1|    1|
%           12|    0|     1|    1|
%           32|    1|     1|    1|
The parsing and search process was conducted on the 245,243 cloned repositories; this yielded 141,400 SATD/fixed pair of method bodies and were stored into a database table.
For each pair we collected the following information:
\begin{itemize}
    \item Pattern: the word or sentence that identified the SATD.
    \item Commit message: the commit message that removed the SATD.
    \item Commit hash id.
    \item Repository name and url.
    \item The body of the method before the commit (affected with SATD) and the body after the commit (fixed).
\end{itemize}

\emph{continue describing the parsed information and the subsequent manual and automatic filtering ...}

%4.3
\section{Replication Package}
%Replication Package [A link to a repo with all data and code]
A replication package is available on GitHub:
\begin{itemize}
    \item The source code for repository mining and deep learning model \footnote{https://github.com/simonegiacomelli/code2vec-satd-classifier} 
    \item Two Postgresql database backups \footnote{https://github.com/simonegiacomelli/code2vec-satd-classifier-dataset } containing:
    \begin{itemize}
\item The dataset with the mined GitHub repository urls and all the method bodies collected
\item The Optuna database containing the data of the distributed experiment (i.e. hyperparameters tuning) with all TensorFlow outputs 
\end{itemize}
\end{itemize}

\chapter{Results Discussion}

%\dots results discussion \dots 

%5.1
\section{Quantitative Results}
Thise section reports the results for {RQ$_1$} and {RQ$_2$}.

%Report precision/recall for different confidence thresholds
\textbf{RQ$_1$} \textit{How does our approach perform in detecting technical debt in unseen methods source code?} 

Table \ref{tbl:rq1} reports the precision, recall, accuracy and F1-score for twelve experiments on different dataset filtered by token count. The results using the smaller dataset, with tokens count less than 50 (dataset-50), shows the best precision (71\%), the second highest recall (62\%) and the best F1-score (66\%). The lowest F1-score (49\%) and lowest precision (42\%) are found with dataset-300. We notice that the second best precision (67\%) and the second highest F1-score comes with the largest dataset-600; this may lead to think that the size of the snippet influence the accuracy of the model only up to a point: in-fact we observe that the accuracy for larger dataset than 200 tokens remains roughly the same.

%https://www.tablesgenerator.com/#
\begin{table}[h!]
\centering
\caption{Twelve experiments on different snippet sizes.\label{tbl:rq1}}
\begin{tabular}{|r|r|r|r|r|r|}
\hline
\multicolumn{1}{|l|}{\#Tokens $< x$} &
  \multicolumn{1}{l|}{\#Test samples} &
  \multicolumn{1}{l|}{Prec.} &
  \multicolumn{1}{l|}{Recall} &
  \multicolumn{1}{l|}{Accuracy} &
  \multicolumn{1}{l|}{F1-score} \\ \hline
50  & 2826  & 71\% & 62\% & 64\% & 66\% \\ %run_id 6
100 & 8206  & 60\% & 62\% & 62\% & 61\% \\
150 & 12612 & 51\% & 63\% & 60\% & 56\% \\
200 & 15940 & 51\% & 60\% & 58\% & 55\% \\ %run_id 7
250 & 18518 & 47\% & 61\% & 58\% & 53\% \\
300 & 20346 & 42\% & 60\% & 57\% & 49\% \\
350 & 21712 & 53\% & 58\% & 57\% & 56\% \\
400 & 22782 & 60\% & 58\% & 58\% & 59\% \\ %run_id 40
450 & 23620 & 59\% & 57\% & 57\% & 58\% \\
500 & 24276 & 61\% & 57\% & 58\% & 59\% \\
550 & 24812 & 56\% & 58\% & 58\% & 57\% \\
600 & 25250 & 67\% & 56\% & 58\% & 61\% \\
\hline
\end{tabular}
\end{table}


\textbf{RQ$_2$} \textit{What is the correlation between the prediction confidence level and the accuracy of the model?}

We take two experiments from RQ$_1$ (dataset-50 and dataset-200) and analyze how the confidence level affects the quality metrics of the predictions. We observe that for the experiment in table \ref{tbl:rq2_token_50}, when filtering for a confidence level greater than 0.9 we reduce the coverage by 44\% gaining on precision from 71\% to 78\%; the excluded samples shows their effect also on the recall that goes from 62\% to 52\%. 
Table \ref{tbl:rq2_token_200} shows a different drop in coverage; with confidence greater than 0.9 the test set covered is about 2\%, the precision is high as 99\% and the recall drops to 10\%, both due to the (correct and incorrect) discarded predictions.
 
Figure \ref{fig:rq2_box_plot} presents, via box plots, the confidence level divided by class (i.e true positive, true negative, false positive and false negative) for both experiments. The plot shows a much greater confidence when using the smaller dataset: the median for the true positive dataset-50 is 0.97 and 0.56 for dataset-200.

\begin{table}[h!]
\centering
\caption{Experiment `\#Tokens $< 50$' split for prediction confidence.\label{tbl:rq2_token_50}}

\begin{tabular}{|r|r|r|r|r|r|}
\hline
  \multicolumn{1}{|l|}{\#Confidence $> x$} &
  \multicolumn{1}{l|}{\#Test samples} &
  \multicolumn{1}{l|}{Test samples coverage} &
  \multicolumn{1}{l|}{Prec.} &
  \multicolumn{1}{l|}{Recall} &
  \multicolumn{1}{l|}{F1-score} \\ 
\hline
0   & 2826 & 100\% & 71\% & 62\% & 66\% \\
0.6 & 2592 & 92\%  & 72\% & 61\% & 66\% \\
0.7 & 2345 & 83\%  & 74\% & 59\% & 66\% \\
0.8 & 2027 & 72\%  & 76\% & 57\% & 65\% \\
0.9 & 1590 & 56\%  & 78\% & 52\% & 63\% \\
\hline
\end{tabular}
\end{table}

\begin{table}[h!]
\centering
\caption{Experiment `\#Tokens $< 200$' split for prediction confidence.\label{tbl:rq2_token_200}}

\begin{tabular}{|r|r|r|r|r|r|}
\hline
  \multicolumn{1}{|l|}{\#Confidence $> x$} &
  \multicolumn{1}{l|}{\#Test samples} &
  \multicolumn{1}{l|}{Test samples coverage} &
  \multicolumn{1}{l|}{Prec.} &
  \multicolumn{1}{l|}{Recall} &
  \multicolumn{1}{l|}{F1-score} \\ 
\hline
0   & 15940 & 100\% & 51\% & 60\% & 55\% \\
0.6 & 4761  & 30\%  & 61\% & 34\% & 44\% \\
0.7 & 1616  & 10\%  & 79\% & 20\% & 32\% \\
0.8 & 757   & 5\%   & 92\% & 14\% & 25\% \\
0.9 & 349   & 2\%   & 99\% & 10\% & 18\% \\ 
\hline
\end{tabular}
\end{table}

%includere i box plot
\begin{figure}[h!]
 \centering
 \resizebox{\columnwidth}{!}{
  \includegraphics{images/box_plot.png}
 }
 \caption[Prediction confidence level split by class.]{Prediction confidence level split by class.}
    \label{fig:rq2_box_plot}
\end{figure}


%5.2 
\section{Qualitative Results}
%Discuss interesting cases in which your approach succeeds/fails
%, trying to explain what could make the model failing in those cases.

We discuss some qualitative examples where our model prediction succeed and where it fails. Then we explain why it was so.
We focus on high confidence predictions (correct and incorrect) using the following two scenarios, the same discussed in RQ$_2$:
\begin{itemize}
    \item Scenario-1: trained with a dataset composed only of those snippets with token\_count less than 50. The test dataset contains 1,413 sample pairs.
    %6,595 training sample pairs.
    \item Scenario-2: trained with a dataset composed only of those snippets with token\_count less than 200. The test dataset contains 7,970 sample pairs. This was the target of the hyperparameters tuning experiment.
    %37,195 training sample pairs.
\end{itemize}

All test session results are stored into a database with the information related to the prediction: a boolean value indicating if it was correct or incorrect, the confidence rating and some information on the attention vector weights. Each record contains these fields in pairs, one for the SATD and one for the fixed. It contains also the identifier link to trace back to the method source code and all related data.

The following paragraphs explains the findings for specific cases of both schenarios.

\textbf{Correct predictions, Scenario-1.} We queried Scenario-1 filtering for those samples that were correctly predicted with confidence greater than 0.99; the filter was applied to both elements of the pair at the same time, so both SATD/fixed were true positive with a high confidence rating.

Then, we manually went through a few of the 91 query hits, inspecting the source code of the SATD; we found the following interesting recurring cases:
\begin{itemize}
    \item Case-1 Empty exception block %id-125475
    \item Case-2 Magic constant %id-66085
    \item Case-3 Return null %id-57467
\end{itemize}

For each of these cases we extracted the weights of the attention vector and the related AST-paths, for both labels: SATD and fixed\footnote{The appendix section \ref{app:qualitative} contains the clean source code snippets of these three cases, the prediction confidence and the attention weights with the AST-paths}. These weights are sorted in descending ordered; looking to those values, we noticed that, for example, the first AST-path weight was roughly twice than the second. This means that the first AST-path was pivotal for the correct prediction; if the AST-path was decisive, it suggests it should be present in a meaningful way in the training set. After fixing the representation of the AST-path to the same stored in the database, we searched all the occurrences of the first particular path in this session training dataset samples. We also inspected the general distribution of weights and explained the most important AST-paths.


\begin{lstlisting}[caption={Case-1 SATD, verbatim source code}, label={lst:case_empty_exception},language=Java]

public boolean postfire() throws IllegalActionException {
    generateEvents(new ExecEvent(this, 2));
    try {
        Thread.sleep(100);
    } catch (InterruptedException e) {
        // FIXME
    }
    return super.postfire();
}

\end{lstlisting}

\begin{lstlisting}[caption={Case-1 fixed, verbatim source code}, label={lst:case1fixed},language=Java]
public boolean postfire() throws IllegalActionException {
    generateEvents(new ExecEvent(this, 2));
    try {
        Thread.sleep(100);
    } catch (InterruptedException e) {
        throw new InternalErrorException("Error with " + "sleeping thread in postfire");
    }
    return super.postfire();
}
\end{lstlisting}


\textbf{Case-1.} Using the first AST-path as a filter on the training dataset, we found three other cases labeled as SATD similar to this instance (see listing \ref{lst:case_empty_exception}). Then we inspected the fixed counterpart of this pair and found that the attention vector was giving roughly 45\% of the importance to the first three AST-paths; all of them has one end of the path in the \textit{throw new Exception} statement inside the \textit{catch} block (i.e. the fixed part of the snippet, see listing \ref{lst:case1fixed}). This could indicate that the model actually learned how to detect this particular technical debt with meaningful discriminating factors.


\textbf{Case-2.} `Magic constant' refers to the anti-pattern of using numbers directly in source code. In detail, Case-2 was found guilty for using the constant `302' to indicate the HTTP result code for \textit{StatusFound}. We inspected the first five AST-paths and searched for them throughout the training samples: we found 21 SATD with such paths. Four out of them were real `Magic constant' SATD, four were SATD related to a call to \textit{System.exit(0)} and the remaining were rich with numeric literals but not true `Magic constant' SATD. In other words, the positive element of this case was correctly classified not entirely for the right reasons.



\begin{lstlisting}[caption={Correctly detected SATD, verbatim Case-2}, label={lst:case_magic_constant},language=Java]

static void httpRedirect(final Exchange exchange, final String uri) {
    // FIXME: this constant should in HTTP package?
    httpResponse(exchange, 302);
    exchange.response.getHeaders().add(LocationHeader.NAME, uri);
}

\end{lstlisting}


\textbf{Case-3.} Returning null from a function is often associated with a bad smell. We went through the training samples with a common AST-path from Case-3 and we found three hit of SATD for a similar `return null'. We inspected other similar snippets to this case from Scenario-1 and found interesting different outcomes. We noticed methods containing a `return null' were correctly classified as negative (i.e. fixed): the code of such methods was using the arguments of the function before, eventually returning null; this changed the attention vector weights far from the `return null' statement. 


\begin{lstlisting}[caption={Case-3 SATD, verbatim source code}, label={lst:case3satd},language=Java]
public ItemStack constructTool(ItemStack rod, ItemStack... materials) {
    // FIXME: 1.11
    if (GemsConfig.TOOL_DISABLE_AXE)
        return null;
    return ToolHelper.constructTool(this, rod, materials);
}
\end{lstlisting}

\begin{lstlisting}[caption={Case-3 fixed, verbatim source code}, label={lst:case3fixed},language=Java]
public ItemStack constructTool(ItemStack rod, ItemStack... materials) {
    if (GemsConfig.TOOL_DISABLE_AXE)
        return ItemStack.EMPTY;
    return ToolHelper.constructTool(this, rod, materials);
}
\end{lstlisting}

\textbf{Wrong predictions, Scenario-1.}
The query yielded only one % id-43338
high confidence wrong prediction in contrast to the previous analysis where we had 91 correct high confidence predictions. Listing \ref{lst:scenario_1_wrong} contains both verbatim elements of the sample pair.


\begin{lstlisting}[caption={Scenario-1 wrong predictions, verbatim source code}, label={lst:scenario_1_wrong},language=Java]
//wrongly predicted as fixed
public String getClientID() {
    // fixme this will only work for 0-10 connections
    // In 0-8 there is an explicit ClientID property that is != Principal.
    return getPrincipal().getName();
}

//wrongly predicted as SATD
public String getClientID() {
    return getConnection().getClientId();
}
\end{lstlisting}

First we observe that the two snippets have the same AST structure: they share the same signature and returns the value coming from two consecutive invocations. Thus, the AST-paths (excluding the value leaves) will be identical. 
We analyzed the presence of such paths into the training samples. 
First we start with the actual SATD element (predicted as fixed), the easiest to explain; we traced the first two AST-paths and found they recur most often in the fixed training samples: 15+44 (59) fixed occurrences versus 8+19 (27) SATD occurrences. 
Second, we explain the wrong prediction for the actual fixed element; this is more interesting because it has the same rough balance of before but leading to a different prediction. In-fact, the first two AST-paths have the following recurrence: 137+516 (653) fixed and 75+266 (341) SATD. Why did it predict with a SATD label and not a fixed? The answer lies in the number of samples that use both AST-paths at the same time: 75 SATD and 39 fixed. This ratio inversion was not present in the counterpart prediction above; the model took this into account as the dominant factor for the SATD prediction. 
%predicted:fixed actual:satd
%
% AST-path-1 50% string,1604992453,getprincipal
% satd  0 hit string,1604992453,getprincipal
% satd  1 hit string,1604992453,%
% satd  8 hit %,1604992453,%
% fixed 0 hit string,1604992453,getprincipal
% fixed 2 hit string,1604992453,%
% fixed 15 hit %,1604992453,%
%
% AST-path-2 22% METHOD_NAME,967193662,getprincipal
% satd  0 hit METHOD_NAME,967193662,getprincipal
% satd 19 hit METHOD_NAME,967193662,%
% satd 19 hit %,967193662,%
% fixed 0 hit METHOD_NAME,967193662,getprincipal
% fixed 44 hit METHOD_NAME,967193662,%
% fixed 44 hit %,967193662,%
%
% AST-path 1 & 2 1604992453&967193662
% satd 8 hit
% fixed 5 hit

%predicted:satd actual:fixed
% AST-path-1 32% string,270643758,getclientid
% satd  0 hit string,270643758,getclientid
% satd 21 hit string,270643758,
% satd 75 hit ,270643758,
% fixed 0 hit string,270643758,getclientid
% fixed 50 hit string,270643758,
% fixed 137 hit ,270643758,
%
% AST-path-2 23% METHOD_NAME,713917671,getclientid
% satd  0 hit METHOD_NAME,713917671,getclientid
% satd  266 hit METHOD_NAME,713917671,
% satd  266 hit 
% fixed 0 hit METHOD_NAME,713917671,getclientid
% fixed 516 hit METHOD_NAME,713917671,
% fixed 516 hit 
% 
% AST-path 1&2 270643758&713917671
% satd  75
% fixed 39
%
% AST-path-3 19% getconnection,1961396084,getclientid
% satd  0 hit getconnection,1961396084,getclientid 
% satd  0/0 hit *getconnection*,1961396084,*getclientid*
% satd  155 hit 1961396084
% fixed 0 hit getconnection,1961396084,getclientid 
% fixed 0/0 hit *getconnection*,1961396084,*getclientid*
% fixed 227 hit 1961396084
%
% AST-path-4 16% string,1604992453,getconnection
% satd  y hit 
% satd  y hit 
% satd  y hit 
% fixed y hit 
% fixed y hit 
% fixed y hit 

\textbf{Correct predictions, Scenario-2.}
 Querying Scenario-2 with confidence threshold set to 0.99 gave no hits; we lowered it to 0.87 and obtained ten sample pairs. This was probably due to Scenario-2 having a broader dataset that leads to a better generalization but brings in more noise. 
 Seven out of ten correct predictions were `return null' cases. We will explain one interesting case taken between the remaining three.


\begin{lstlisting}[caption={Scenario-2 correct predictions, verbatim source code}, label={lst:scenario_2_correct},language=Java]
public boolean contains(String s) {
    // FIXME
    return false;
}

public boolean contains(String s) {
    return containsHelper(s, root);
}
\end{lstlisting}

The model correctly labeled as SATD the method composed only by `return false'; this is a consequence of not using the argument `s' of the method. On the other hand, the fixed counterpart was found to employ AST-paths involving the argument `s' in the evaluation of the return value.

\textbf{Wrong predictions, Scenario-2.}
We lowered the confidence ration threshold to 0.7 and got four wrong prediction samples pair. This means that the model is less sure about the results probably due to the noisy dataset.
The actual fixed, wrongly predicted as SATD, are mainly `return null' that usually are actual SATD. In these cases, the fixes introduced the (normally) smelly code.
\chapter{Threats to Validity}

\section{Construct validity}
Threats to \textit{construct validity} concern the relation between the theory and the observation. In other words, the threat is whether the measurements performed really represent what is investigated according to our research questions. In this study we mined the dataset from scratch, which is a third degree type of data \cite{runeson2009guidelines}, and we are aware of the threats explained in the following paragraphs. 

The SATD detection relies on the keyword pattern matching proposed by Potdar and Shihab \cite{potdar2014exploratory}. Such a heuristic can introduce imprecisions in the correct identification of SATD in code comments. It is estimated that the original pattern list is likely to produce $\sim$25\% of false positive SATD \cite{bavota2016large}. To diminish this issue we manually verified more than one hundred random samples and made sure to exclude some keywords that were repeatedly found to produce many false positives.
There might be better strategies for SATD identification. Instead of keyword matching, other researches employ natural language processing (NLP) \cite{maldonado2017using} or deep learning \cite{wang2020detecting}. 

Stale comments with matching keywords in it are detected as SATD but are actually harmless non-SATD comments; in our procedure then, we locate the commit that removes the SATD comment and we identify the `fixed' code. We are aware that this leads to the introduction of a false positives in the training dataset. However, it is shown \cite{bavota2016large} that such cases only represent less than 10\% of the overall SATD instances. Thus, the impact on our findings is limited.

As observed in multiple studies there are many kinds of SATD \cite{alves2014towards} \cite{maldonado2015detecting}; specifically, self-admitted design debts need a broader context to be identified than the single method body. This information is simply not present in the boundaries of the snippet, so the model is hindered in learning this type of SATD with the code representation we use in this research. In other words, we might have (some) code snippets labeled as SATD but such information is not fully shown by the features extracted from the code.

Also, possible imprecisions might be introduced due to errors in the implementation of the tool we wrote to create the dataset. We wrote automated tests to ensure the correct behaviour of our tool and all the source code is available in the replication package.

\section{Internal validity}
Threats to \textit{internal validity} concern external factors we did not consider that could affect the variables and the relations being investigated. To avoid implementation errors, we carefully reviewed our hyperparameter settings. Our grid search did not exhaust the search space but covered a reasonable interval in the hyperparameter interval window.

\section{External  validity}
Threats to \textit{external validity} concern the generalisation of results. Although we mined a large number of projects (245,243), other systems should be analysed to support our conclusions. This is especially needed due to the fact that (i) all the projects subject of our study are written in Java, thus calling for the need of analysing software projects written in other programming languages, and (ii) we limited our analysis to openly available GitHub projects ignoring industrial systems.
\chapter{Conclusion}

\dots conclusion \dots 

%%%%%%% commented appendix

\appendix

\chapter{Additional material}

\section{}

\begin{lstlisting}[caption={61 patterns for SATD detection}, label={lst:patterns61}]
                # a line that starts with an hash is ignored
                hack
                retarded
                at a loss
                stupid
                remove this code
                ugly
                take care
                something's gone wrong
                nuke
                is problematic
                may cause problem
                hacky
                unknown why we ever experience this
                treat this as a soft error
                silly
                workaround for bug
                kludge
                fixme
                this isn't quite right
                trial and error
                give up
                this is wrong
                hang our heads in shame
                temporary solution
                causes issue
                something bad is going on
                cause for issue
                this doesn't look right
                is this next line safe
                this indicates a more fundamental problem
                temporary crutch
                this can be a mess
                this isn't very solid
                this is temporary and will go away
                is this line really safe
                #there is a problem
                some fatal error
                something serious is wrong
                don't use this
                get rid of this
                doubt that this would work
                this is bs
                give up and go away
                risk of this blowing up
                just abandon it
                prolly a bug
                probably a bug
                hope everything will work
                toss it
                barf
                something bad happened
                fix this crap
                yuck
                certainly buggy
                remove me before production
                you can be unhappy now
                this is uncool
                #bail out
                it doesn't work yet
                crap
                inconsistency
                abandon all hope
                kaboom
\end{lstlisting}

\section{Qualitative results}\label{app:qualitative} 

\subsection{Case-1}

\begin{lstlisting}[basicstyle=\tiny,caption={Case-1 SATD}, label={},language=Java,breaklines=true,
  postbreak=\mbox{\textcolor{red}{$\hookrightarrow$}\space}]
/*
Prediction:	satd
Actual:	satd
	(0.994157) predicted: ['satd']
	(0.005767) predicted: ['fixed']
Attention:
0.055485	context: sleep,(NameExpr3)^(MethodCallExpr)^(ExpressionStmt)^(BlockStmt)^(TryStmt)^(BlockStmt)_(ReturnStmt)_(MethodCallExpr0)_(NameExpr2),postfire
0.047134	context: interruptedexception,(ClassOrInterfaceType1)^(Parameter)^(CatchClause)^(TryStmt)^(BlockStmt)_(ReturnStmt)_(MethodCallExpr0)_(NameExpr2),postfire
0.042867	context: 2,(IntegerLiteralExpr2)^(ObjectCreationExpr1)^(MethodCallExpr)^(ExpressionStmt)^(BlockStmt)_(ReturnStmt)_(MethodCallExpr0)_(NameExpr2),postfire
0.038576	context: e,(VariableDeclaratorId0)^(Parameter)^(CatchClause)^(TryStmt)^(BlockStmt)_(ReturnStmt)_(MethodCallExpr0)_(NameExpr2),postfire
0.036747	context: 100,(IntegerLiteralExpr2)^(MethodCallExpr)^(ExpressionStmt)^(BlockStmt)^(TryStmt)^(BlockStmt)_(ReturnStmt)_(MethodCallExpr0)_(NameExpr2),postfire
0.036469	context: thread,(NameExpr0)^(MethodCallExpr)^(ExpressionStmt)^(BlockStmt)^(TryStmt)^(BlockStmt)_(ReturnStmt)_(MethodCallExpr0)_(NameExpr2),postfire
0.036007	context: this,(ThisExpr1)^(ObjectCreationExpr1)^(MethodCallExpr)^(ExpressionStmt)^(BlockStmt)_(ReturnStmt)_(MethodCallExpr0)_(NameExpr2),postfire
0.029727	context: illegalactionexception,(ClassOrInterfaceType2)^(MethodDeclaration)_(BlockStmt)_(ExpressionStmt)_(MethodCallExpr0)_(ObjectCreationExpr)_(IntegerLiteralExpr2),2
0.027548	context: this,(ThisExpr1)^(ObjectCreationExpr1)_(IntegerLiteralExpr2),2
0.025591	context: METHOD_NAME,(NameExpr1)^(MethodDeclaration)_(BlockStmt)_(ExpressionStmt)_(MethodCallExpr0)_(ObjectCreationExpr)_(IntegerLiteralExpr2),2
*/

public class Wrapper{
    public boolean satd() throws IllegalActionException {
        generateEvents(new ExecEvent(this, 2));
        try {
            Thread.sleep(100);
        } catch (InterruptedException e) {
        }
        return super.postfire();
    }
}
\end{lstlisting}

\begin{lstlisting}[basicstyle=\tiny,caption={Case-1 fixed}, label={},language=Java,breaklines=true,
  postbreak=\mbox{\textcolor{red}{$\hookrightarrow$}\space}]
/*
Prediction:	fixed
Actual:	fixed
	(0.999927) predicted: ['fixed']
Attention:
0.162827	context: illegalactionexception,(ClassOrInterfaceType2)^(MethodDeclaration)_(BlockStmt)_(TryStmt)_(CatchClause)_(BlockStmt)_(ThrowStmt)_(ObjectCreationExpr)_(ClassOrInterfaceType0),internalerrorexception
0.162034	context: e,(VariableDeclaratorId0)^(Parameter)^(CatchClause)_(BlockStmt)_(ThrowStmt)_(ObjectCreationExpr)_(ClassOrInterfaceType0),internalerrorexception
0.146301	context: interruptedexception,(ClassOrInterfaceType1)^(Parameter)^(CatchClause)_(BlockStmt)_(ThrowStmt)_(ObjectCreationExpr)_(ClassOrInterfaceType0),internalerrorexception
0.072005	context: METHOD_NAME,(NameExpr1)^(MethodDeclaration)_(BlockStmt)_(TryStmt)_(CatchClause)_(BlockStmt)_(ThrowStmt)_(ObjectCreationExpr)_(ClassOrInterfaceType0),internalerrorexception
0.023477	context: sleep,(NameExpr3)^(MethodCallExpr)^(ExpressionStmt)^(BlockStmt)^(TryStmt)^(BlockStmt)_(ReturnStmt)_(MethodCallExpr0)_(NameExpr2),postfire
0.019943	context: interruptedexception,(ClassOrInterfaceType1)^(Parameter)^(CatchClause)^(TryStmt)^(BlockStmt)_(ReturnStmt)_(MethodCallExpr0)_(NameExpr2),postfire
0.018138	context: 2,(IntegerLiteralExpr2)^(ObjectCreationExpr1)^(MethodCallExpr)^(ExpressionStmt)^(BlockStmt)_(ReturnStmt)_(MethodCallExpr0)_(NameExpr2),postfire
0.016322	context: e,(VariableDeclaratorId0)^(Parameter)^(CatchClause)^(TryStmt)^(BlockStmt)_(ReturnStmt)_(MethodCallExpr0)_(NameExpr2),postfire
0.015548	context: 100,(IntegerLiteralExpr2)^(MethodCallExpr)^(ExpressionStmt)^(BlockStmt)^(TryStmt)^(BlockStmt)_(ReturnStmt)_(MethodCallExpr0)_(NameExpr2),postfire
0.015431	context: thread,(NameExpr0)^(MethodCallExpr)^(ExpressionStmt)^(BlockStmt)^(TryStmt)^(BlockStmt)_(ReturnStmt)_(MethodCallExpr0)_(NameExpr2),postfire
*/

public class Wrapper{
    public boolean fixed() throws IllegalActionException {
        generateEvents(new ExecEvent(this, 2));
        try {
            Thread.sleep(100);
        } catch (InterruptedException e) {
            throw new InternalErrorException("--##string##--" + "--##string##--");
        }
        return super.postfire();
    }
}
\end{lstlisting}

\subsection{Case-2}

\begin{lstlisting}[basicstyle=\tiny,caption={Case-2 SATD}, label={},language=Java,breaklines=true,
  postbreak=\mbox{\textcolor{red}{$\hookrightarrow$}\space}]
/*
Prediction:	satd
Actual:	satd
	(0.999226) predicted: ['satd']
	(0.000668) predicted: ['fixed']
Attention:
0.065232	context: exchange,(ClassOrInterfaceType1)^(Parameter)^(MethodDeclaration)_(BlockStmt)_(ExpressionStmt)_(MethodCallExpr0)_(IntegerLiteralExpr2),302
0.058992	context: exchange,(NameExpr1)^(MethodCallExpr)_(IntegerLiteralExpr2),302
0.052524	context: string,(ClassOrInterfaceType1)^(Parameter)^(MethodDeclaration)_(BlockStmt)_(ExpressionStmt)_(MethodCallExpr0)_(IntegerLiteralExpr2),302
0.046421	context: exchange,(VariableDeclaratorId0)^(Parameter)^(MethodDeclaration)_(BlockStmt)_(ExpressionStmt)_(MethodCallExpr0)_(IntegerLiteralExpr2),302
0.038198	context: uri,(VariableDeclaratorId0)^(Parameter)^(MethodDeclaration)_(BlockStmt)_(ExpressionStmt)_(MethodCallExpr0)_(IntegerLiteralExpr2),302
0.036755	context: exchange,(NameExpr0)^(FieldAccessExpr0)^(MethodCallExpr0)^(MethodCallExpr)_(FieldAccessExpr2)_(NameExpr0),locationheader
0.030503	context: 302,(IntegerLiteralExpr2)^(MethodCallExpr)^(ExpressionStmt)^(BlockStmt)_(ExpressionStmt)_(MethodCallExpr0)_(MethodCallExpr0)_(FieldAccessExpr0)_(NameExpr0),exchange
0.029021	context: 302,(IntegerLiteralExpr2)^(MethodCallExpr)^(ExpressionStmt)^(BlockStmt)_(ExpressionStmt)_(MethodCallExpr0)_(FieldAccessExpr2)_(NameExpr0),locationheader
0.025117	context: exchange,(NameExpr1)^(MethodCallExpr)^(ExpressionStmt)^(BlockStmt)_(ExpressionStmt)_(MethodCallExpr0)_(FieldAccessExpr2)_(NameExpr0),locationheader
0.021635	context: httpresponse,(NameExpr3)^(MethodCallExpr)^(ExpressionStmt)^(BlockStmt)_(ExpressionStmt)_(MethodCallExpr0)_(MethodCallExpr0)_(FieldAccessExpr0)_(NameExpr0),exchange
*/

public class Wrapper{
    static void satd(final Exchange exchange, final String uri) {
        httpResponse(exchange, 302);
        exchange.response.getHeaders().add(LocationHeader.NAME, uri);
    }
}
\end{lstlisting}

\begin{lstlisting}[basicstyle=\tiny,caption={Case-2 fixed}, label={},language=Java,breaklines=true,
  postbreak=\mbox{\textcolor{red}{$\hookrightarrow$}\space}]
/*
Prediction:	fixed
Actual:	fixed
	(0.999886) predicted: ['fixed']
	(0.000081) predicted: ['satd']
Attention:
0.142238	context: response,(ClassOrInterfaceType0)^(VariableDeclarationExpr)_(VariableDeclarator)_(MethodCallExpr1)_(FieldAccessExpr1)_(NameExpr2),found
0.140574	context: response,(VariableDeclaratorId0)^(VariableDeclarator)_(MethodCallExpr1)_(FieldAccessExpr1)_(NameExpr2),found
0.125663	context: METHOD_NAME,(NameExpr1)^(MethodDeclaration)_(BlockStmt)_(ExpressionStmt)_(VariableDeclarationExpr)_(VariableDeclarator)_(MethodCallExpr1)_(FieldAccessExpr1)_(NameExpr2),found
0.032170	context: response,(VariableDeclaratorId0)^(VariableDeclarator)_(MethodCallExpr1)_(FieldAccessExpr1)_(NameExpr0),status
0.032053	context: status,(NameExpr0)^(FieldAccessExpr1)_(NameExpr2),found
0.021654	context: response,(ClassOrInterfaceType0)^(VariableDeclarationExpr)^(ExpressionStmt)^(BlockStmt)_(ExpressionStmt)_(MethodCallExpr0)_(FieldAccessExpr2)_(NameExpr0),locationheader
0.020603	context: found,(NameExpr2)^(FieldAccessExpr1)^(MethodCallExpr)^(VariableDeclarator)^(VariableDeclarationExpr)^(ExpressionStmt)^(BlockStmt)_(ReturnStmt)_(NameExpr0),response
0.019897	context: response,(ClassOrInterfaceType0)^(VariableDeclarationExpr)_(VariableDeclarator)_(MethodCallExpr1)_(FieldAccessExpr1)_(NameExpr0),status
0.018418	context: status,(NameExpr0)^(FieldAccessExpr1)^(MethodCallExpr)^(VariableDeclarator)^(VariableDeclarationExpr)^(ExpressionStmt)^(BlockStmt)_(ReturnStmt)_(NameExpr0),response
0.017697	context: string,(ClassOrInterfaceType1)^(Parameter)^(MethodDeclaration)_(BlockStmt)_(ExpressionStmt)_(MethodCallExpr0)_(FieldAccessExpr2)_(NameExpr0),locationheader
*/

public class Wrapper{
    static Response fixed(final String uri) {
        Response response = httpResponse(Status.FOUND);
        response.getHeaders().add(LocationHeader.NAME, uri);
        return response;
    }
}
\end{lstlisting}


\subsection{Case-3}

\begin{lstlisting}[basicstyle=\tiny,caption={Case-3 SATD}, label={},language=Java,breaklines=true,
  postbreak=\mbox{\textcolor{red}{$\hookrightarrow$}\space}]
/*
Prediction:	satd
Actual:	satd
	(0.999509) predicted: ['satd']
	(0.000419) predicted: ['fixed']
Attention:
0.132759	context: null,(NullLiteralExpr0)^(ReturnStmt)^(IfStmt)^(BlockStmt)_(ReturnStmt)_(MethodCallExpr0)_(NameExpr5),constructtool
0.064306	context: null,(NullLiteralExpr0)^(ReturnStmt)^(IfStmt)^(BlockStmt)_(ReturnStmt)_(MethodCallExpr0)_(NameExpr4),materials
0.060350	context: null,(NullLiteralExpr0)^(ReturnStmt)^(IfStmt)^(BlockStmt)_(ReturnStmt)_(MethodCallExpr0)_(ThisExpr2),this
0.050614	context: gemsconfig,(NameExpr0)^(FieldAccessExpr)^(IfStmt)_(ReturnStmt)_(NullLiteralExpr0),null
0.041519	context: tooldisableaxe,(NameExpr2)^(FieldAccessExpr)^(IfStmt)_(ReturnStmt)_(NullLiteralExpr0),null
0.036061	context: tooldisableaxe,(NameExpr2)^(FieldAccessExpr)^(IfStmt)^(BlockStmt)_(ReturnStmt)_(MethodCallExpr0)_(NameExpr5),constructtool
0.032750	context: materials,(VariableDeclaratorId0)^(Parameter)^(MethodDeclaration)_(BlockStmt)_(IfStmt)_(FieldAccessExpr0)_(NameExpr2),tooldisableaxe
0.026796	context: itemstack,(ClassOrInterfaceType1)^(Parameter)^(MethodDeclaration)_(BlockStmt)_(IfStmt)_(FieldAccessExpr0)_(NameExpr2),tooldisableaxe
0.025070	context: rod,(VariableDeclaratorId0)^(Parameter)^(MethodDeclaration)_(BlockStmt)_(IfStmt)_(FieldAccessExpr0)_(NameExpr2),tooldisableaxe
0.020929	context: tooldisableaxe,(NameExpr2)^(FieldAccessExpr)^(IfStmt)^(BlockStmt)_(ReturnStmt)_(MethodCallExpr0)_(NameExpr4),materials
*/

public class Wrapper{
    @Override
    public ItemStack satd(ItemStack rod, ItemStack... materials) {
        if (GemsConfig.TOOL_DISABLE_AXE)
            return null;
        return ToolHelper.constructTool(this, rod, materials);
    }
}
\end{lstlisting}

\begin{lstlisting}[basicstyle=\tiny,caption={Case-3 fixed}, label={},language=Java,breaklines=true,
  postbreak=\mbox{\textcolor{red}{$\hookrightarrow$}\space}]
/*
Prediction:	fixed
Actual:	fixed
	(0.996488) predicted: ['fixed']
	(0.003202) predicted: ['satd']
Attention:
0.098242	context: tooldisableaxe,(NameExpr2)^(FieldAccessExpr)^(IfStmt)_(ReturnStmt)_(FieldAccessExpr0)_(NameExpr0),itemstack
0.042913	context: empty,(NameExpr2)^(FieldAccessExpr)^(ReturnStmt)^(IfStmt)^(BlockStmt)_(ReturnStmt)_(MethodCallExpr0)_(NameExpr5),constructtool
0.037714	context: gemsconfig,(NameExpr0)^(FieldAccessExpr)^(IfStmt)_(ReturnStmt)_(FieldAccessExpr0)_(NameExpr0),itemstack
0.033466	context: itemstack,(NameExpr0)^(FieldAccessExpr)^(ReturnStmt)^(IfStmt)^(BlockStmt)_(ReturnStmt)_(MethodCallExpr0)_(NameExpr5),constructtool
0.032388	context: empty,(NameExpr2)^(FieldAccessExpr)^(ReturnStmt)^(IfStmt)^(BlockStmt)_(ReturnStmt)_(MethodCallExpr0)_(NameExpr4),materials
0.031300	context: tooldisableaxe,(NameExpr2)^(FieldAccessExpr)^(IfStmt)^(BlockStmt)_(ReturnStmt)_(MethodCallExpr0)_(NameExpr5),constructtool
0.028426	context: materials,(VariableDeclaratorId0)^(Parameter)^(MethodDeclaration)_(BlockStmt)_(IfStmt)_(FieldAccessExpr0)_(NameExpr2),tooldisableaxe
0.027018	context: materials,(VariableDeclaratorId0)^(Parameter)^(MethodDeclaration)_(BlockStmt)_(IfStmt)_(ReturnStmt)_(FieldAccessExpr0)_(NameExpr0),itemstack
0.024788	context: empty,(NameExpr2)^(FieldAccessExpr)^(ReturnStmt)^(IfStmt)^(BlockStmt)_(ReturnStmt)_(MethodCallExpr0)_(ThisExpr2),this
0.024491	context: itemstack,(NameExpr0)^(FieldAccessExpr)^(ReturnStmt)^(IfStmt)^(BlockStmt)_(ReturnStmt)_(MethodCallExpr0)_(NameExpr4),materials
*/

public class Wrapper{
    @Override
    public ItemStack fixed(ItemStack rod, ItemStack... materials) {
        if (GemsConfig.TOOL_DISABLE_AXE)
            return ItemStack.EMPTY;
        return ToolHelper.constructTool(this, rod, materials);
    }
}
\end{lstlisting}


% \appendix %optional, use only if you have an appendix
% \chapter{Some retarded material}
% \section{It's over\dots}
% \lipsum 

\nocite{*}
\backmatter

%%%%%%% commented glossary
%\chapter{Glossary} %optional

%\bibliographystyle{alpha}
%\bibliographystyle{dcu}
%\bibliographystyle{plainnat}
%\bibliography{biblio}



\printbibliography
%\cleardoublepage
%\theindex %optional, use only if you have an index, must use
%\makeindex in the preamble
\end{document}