\chapter{Introduction}
Technical Debt (TD) is a metaphor that has its roots in the financial field. In software engineering, a TD is contracted when a workaround or shortcut is taken during code implementation.
Choosing an easy and quick solution over a slower and more appropriate one can save time to deliver the artifact faster in the short run. However, it has many downsides in the long run. Many studies have shown that further work on the affected parts will be more expensive and time consuming than work on clean and healthy code \cite{tom2013exploration} \cite{allman2012managing} \cite{guo2016exploring} \cite{besker2018embracing} \cite{cunningham1992wycash}.
There are also indirect effects to contemplate: the software could misbehave in the operating domain; unexpected output or wrong behavior can cause damages and increased costs \cite{guo2011tracking}.
% \\
% TD do not only appear in software projects but can also be found in many layers of a technology stack: for example, delaying an hardware upgrade or a maintenance can give an immediate benefit of less downtime or financial savings, but an increased cost for future unexpected downtime or failures \cite{allman2012managing}.
%
Developers or managers can accumulate TD, knowingly or unknowingly, because of strict deadlines, limited resources available or just plain laziness \cite{hinsen2015technical,allman2012managing}. Cunningham, who coined the technical debt metaphor, writes: `A little debt speeds development so long as it is paid back promptly with a rewrite' \cite{cunningham1992wycash}. Cunningham implies that one could benefit from a small amount of TD but it should be paid back as soon as possible. 
%
%TD can arise from intentional and unintentional decision, e.g. an inexperienced person could contract it without being aware of what is happening \cite{hinsen2015technical}; In both cases it's often done for saving the limited available resources and shortening time-to-market \cite{tom2012consolidated}. For example, startups are highly pressured to quickly test products and ideas in order to save capitals and be faster than competitors.
%Besker et al. studied how startups incur in TD, which are the factors, challenges and benefits of intentional acquisition of TD; two of the regulating factors found by the authors are the experience of the developers and the software knowledge of the founders \cite{besker2018embracing}.
In contrast to the beneficial viewpoint of TD, Ron Jeffries argues that the metaphor could be `Perhaps too gentle', because it highlights the wise aspect of the choice of contracting a debt; the problem is that people also take debt unwisely. 
Technical debt benefits a software project as long as it is handled before the bigger long term cost is realized.
%
It is generally accepted by the community that technical debt needs to be managed otherwise bad things will happen \cite{guo2016exploring} \cite{hinsen2015technical} \cite{allman2012managing} \cite{martini2018technical}. 

This is why it is important to detect technical debt: avoid additional costs and unwelcome consequences. 
%There are multiple approaches to detect TD:
%The literature shows multiple approaches to automatically detect TD.  

When developers are aware that they introduce technical debt in the source code, for the many reasons we mentioned above, they can decide to leave a code comment to document what they are doing; this is called self-admitted technical debt (SATD) \cite{potdar2014exploratory}. In other words, a SATD is a written confession left as a testimony that there is something `not-quite-right'.
% approaches to detect SATD

%1.1

\section{Objectives and Results}
%Once the context is clear, I will explain the goal of the thesis and summarize the achieved results
There is an extensive literature on techniques to detect technical debt automatically. However, to the best of our knowledge, no study leverages SATD annotations to create a technical debt detector.
We propose a method to use SATD comments to detect technical debt in source code using a deep learning model. 
This thesis describes how to learn a model that classifies code snippets into two categories: TD-free or TD-affected.
To create the dataset needed to learn the model, we processed 245,243 projects commit histories cloned from public GitHub repositories. 
The model internally represents the code snippet as a fixed-length vector. The pre-processing pipeline extracts features from the source code (AST-paths) and they are embedded as code-vectors. Conceptually, this is similar to how word2vec works. The distance between two vectors gives a measure of how similar or dissimilar two snippets are. The last part of the model is a fully connected layer; its purpose is to perform the final binary classification of the code-vector. 

We present a large-scale empirical study to (i) assess the accuracy of our classifier and (ii) research the correlation between the model prediction confidence and the accuracy of the prediction itself. Besides quantitatively evaluating the results, we present a qualitative analysis of concrete high confidence predictions.

% quant results 
We tested the accuracy of the model on two two different datasets with unequal sizes. The results using the smaller dataset, shows 71\% for precision, 62\% recall and 66\% for F1-score. The second best precision (67\%) and the second highest F1-score (61\%) comes with the largest dataset.

%quant results with confidence
The results on the confidence level analysis show interesting results: when filtering for a confidence level greater than 0.9 we measured a gain on precision, from 71\% to 78\%; the excluded samples shows their effect also on the recall that goes from 62\% down to 52\%. When using a dataset with bigger code snippets and confidence greater than 0.9 we discard 98\% of the test predictions; however the precision is high as 99\% and the recall drops to 10\%, both due to the (correct and incorrect) discarded predictions.




% The goal of the thesis is to exploit SATD to train a model that can acquire the capacity to distinguish between TD-free code and code affected by TD.
% Using the comments in open source projects I will identify class methods noted as SATD. Through the vcs commit history I will identify when this comments disappear; the assumption here is that when the comment is removed from the code the SATD has been fixed. Many cases are excluded to minimize the probability of keeping a false positive, i.e. the SATD is not fixed but the comment is removed. For example, the simplest case excluded is when the code is exactly the same and the only change is the SATD comment removal. Another reason of exclusion is when the code is changed too much; in such case it would not be prudent to keep the sample in the dataset.

% The achieved results tell us that it is difficult to predict with high accuracy on methods with big bodies. As the train dataset is limited to shorter and shorter method size, the accuracy grows. 


%1.2
\section{Structure of the Thesis}
The remainder of the manuscript is organized as follows.
\begin{itemize}
 \item \textbf{Chapter 2} presents the state of the art in technical debt and self-admitted technical debt. It explores automatic software bugs identification and introduces the basic concepts of code smells, anti-patterns and the prominent literature on these subjects. The last part examines the most recent researches on machine learning applied to technical debt.
 \item \textbf{Chapter 3} explains how we used deep learning to detect technical debt. There are three main sections: dataset creation, deep learning model and hyperparameter tuning.
 \item \textbf{Chapter 4} presents the research questions and details the design and planning of the study.
 \item \textbf{Chapter 5} presents a quantitative analysis of the results that we achieved. We also present qualitative research and related findings.
 \item \textbf{Chapter 6} discusses  threats to validity. We explain which threats we faced and show how we mitigate them.
 \item \textbf{Chapter 7} concludes the work by summarizing our findings and recommends directions for future work.
\end{itemize}




% \\*******
% \\
% It is useful to list some aspects of TD that form the basis for a tractable model of the phenomenon; many of these aspects comes from the already well known attributes of financial debt. The \emph{principal} is the amount that needs to be payed to 'make things right'. The \emph{interest amount} is the added cost in addition to the principal;  

% In order to better analyze and create a tractable model of TD, Alves et al. identifies three variables \cite{martini2018technical}: 


% Identified variables (3)
% Tracking and managing is important
% Few proposed tracking and managing examples

% there is a cost paying the debt and there are interest costs when efforts are wasted coping with non optimal code.



% Why detecting it is important\\
% \\
% the satd is there
% \\
% it’s undocumented
% \\
% the team is not aware of it
% \\
% \\
% Tom et al. in their systematic literature review studied, among others, the benefits and drawbacks of incurring in TD \cite{tom2012consolidated}; the part we are now interested in is the drawbacks: 
% \\
% Increasing costs over time, such as the amount of effort required to deliver a certain amount of functionality 
% \\
% Work estimation becomes difficult
% \\
% Developer productivity is negatively impacted 
% \\
% Becomes increasingly difficult to repay as decisions are affected by existing debt 
% \\
% Increased risk involved in modifications to the system
% \\
% Change becomes prohibitively expensive to the point of bankruptcy, and a complete rewrite
% and new platform may become necessary
% \\
% Decreased quality in the end product
% \\
% \\
% Ci sono team che risolvo as-you-go, altri hanno strategie per rilevare, identificare e fixare.
% It has been studied that TD [Investigating the Impact of Design Debt on Software Quality]
% E' importante rilevare e monitorare i TD perche' questi aumentano il costo di un progetto software, costo dovuto dall'interesse che si deve pagare in relazione al TD.

% \newpage
